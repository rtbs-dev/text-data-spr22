\documentclass[11pt]{article}

    \usepackage[breakable]{tcolorbox}
    \usepackage{parskip} % Stop auto-indenting (to mimic markdown behaviour)
    
    \usepackage{iftex}
    \ifPDFTeX
    	\usepackage[T1]{fontenc}
    	\usepackage{mathpazo}
    \else
    	\usepackage{fontspec}
    \fi

    % Basic figure setup, for now with no caption control since it's done
    % automatically by Pandoc (which extracts ![](path) syntax from Markdown).
    \usepackage{graphicx}
    % Maintain compatibility with old templates. Remove in nbconvert 6.0
    \let\Oldincludegraphics\includegraphics
    % Ensure that by default, figures have no caption (until we provide a
    % proper Figure object with a Caption API and a way to capture that
    % in the conversion process - todo).
    \usepackage{caption}
    \DeclareCaptionFormat{nocaption}{}
    \captionsetup{format=nocaption,aboveskip=0pt,belowskip=0pt}

    \usepackage[Export]{adjustbox} % Used to constrain images to a maximum size
    \adjustboxset{max size={0.9\linewidth}{0.9\paperheight}}
    \usepackage{float}
    \floatplacement{figure}{H} % forces figures to be placed at the correct location
    \usepackage{xcolor} % Allow colors to be defined
    \usepackage{enumerate} % Needed for markdown enumerations to work
    \usepackage{geometry} % Used to adjust the document margins
    \usepackage{amsmath} % Equations
    \usepackage{amssymb} % Equations
    \usepackage{textcomp} % defines textquotesingle
    % Hack from http://tex.stackexchange.com/a/47451/13684:
    \AtBeginDocument{%
        \def\PYZsq{\textquotesingle}% Upright quotes in Pygmentized code
    }
    \usepackage{upquote} % Upright quotes for verbatim code
    \usepackage{eurosym} % defines \euro
    \usepackage[mathletters]{ucs} % Extended unicode (utf-8) support
    \usepackage{fancyvrb} % verbatim replacement that allows latex
    \usepackage{grffile} % extends the file name processing of package graphics 
                         % to support a larger range
    \makeatletter % fix for grffile with XeLaTeX
    \def\Gread@@xetex#1{%
      \IfFileExists{"\Gin@base".bb}%
      {\Gread@eps{\Gin@base.bb}}%
      {\Gread@@xetex@aux#1}%
    }
    \makeatother

    % The hyperref package gives us a pdf with properly built
    % internal navigation ('pdf bookmarks' for the table of contents,
    % internal cross-reference links, web links for URLs, etc.)
    \usepackage{hyperref}
    % The default LaTeX title has an obnoxious amount of whitespace. By default,
    % titling removes some of it. It also provides customization options.
    \usepackage{titling}
    \usepackage{longtable} % longtable support required by pandoc >1.10
    \usepackage{booktabs}  % table support for pandoc > 1.12.2
    \usepackage[inline]{enumitem} % IRkernel/repr support (it uses the enumerate* environment)
    \usepackage[normalem]{ulem} % ulem is needed to support strikethroughs (\sout)
                                % normalem makes italics be italics, not underlines
    \usepackage{mathrsfs}
    

    
    % Colors for the hyperref package
    \definecolor{urlcolor}{rgb}{0,.145,.698}
    \definecolor{linkcolor}{rgb}{.71,0.21,0.01}
    \definecolor{citecolor}{rgb}{.12,.54,.11}

    % ANSI colors
    \definecolor{ansi-black}{HTML}{3E424D}
    \definecolor{ansi-black-intense}{HTML}{282C36}
    \definecolor{ansi-red}{HTML}{E75C58}
    \definecolor{ansi-red-intense}{HTML}{B22B31}
    \definecolor{ansi-green}{HTML}{00A250}
    \definecolor{ansi-green-intense}{HTML}{007427}
    \definecolor{ansi-yellow}{HTML}{DDB62B}
    \definecolor{ansi-yellow-intense}{HTML}{B27D12}
    \definecolor{ansi-blue}{HTML}{208FFB}
    \definecolor{ansi-blue-intense}{HTML}{0065CA}
    \definecolor{ansi-magenta}{HTML}{D160C4}
    \definecolor{ansi-magenta-intense}{HTML}{A03196}
    \definecolor{ansi-cyan}{HTML}{60C6C8}
    \definecolor{ansi-cyan-intense}{HTML}{258F8F}
    \definecolor{ansi-white}{HTML}{C5C1B4}
    \definecolor{ansi-white-intense}{HTML}{A1A6B2}
    \definecolor{ansi-default-inverse-fg}{HTML}{FFFFFF}
    \definecolor{ansi-default-inverse-bg}{HTML}{000000}

    % commands and environments needed by pandoc snippets
    % extracted from the output of `pandoc -s`
    \providecommand{\tightlist}{%
      \setlength{\itemsep}{0pt}\setlength{\parskip}{0pt}}
    \DefineVerbatimEnvironment{Highlighting}{Verbatim}{commandchars=\\\{\}}
    % Add ',fontsize=\small' for more characters per line
    \newenvironment{Shaded}{}{}
    \newcommand{\KeywordTok}[1]{\textcolor[rgb]{0.00,0.44,0.13}{\textbf{{#1}}}}
    \newcommand{\DataTypeTok}[1]{\textcolor[rgb]{0.56,0.13,0.00}{{#1}}}
    \newcommand{\DecValTok}[1]{\textcolor[rgb]{0.25,0.63,0.44}{{#1}}}
    \newcommand{\BaseNTok}[1]{\textcolor[rgb]{0.25,0.63,0.44}{{#1}}}
    \newcommand{\FloatTok}[1]{\textcolor[rgb]{0.25,0.63,0.44}{{#1}}}
    \newcommand{\CharTok}[1]{\textcolor[rgb]{0.25,0.44,0.63}{{#1}}}
    \newcommand{\StringTok}[1]{\textcolor[rgb]{0.25,0.44,0.63}{{#1}}}
    \newcommand{\CommentTok}[1]{\textcolor[rgb]{0.38,0.63,0.69}{\textit{{#1}}}}
    \newcommand{\OtherTok}[1]{\textcolor[rgb]{0.00,0.44,0.13}{{#1}}}
    \newcommand{\AlertTok}[1]{\textcolor[rgb]{1.00,0.00,0.00}{\textbf{{#1}}}}
    \newcommand{\FunctionTok}[1]{\textcolor[rgb]{0.02,0.16,0.49}{{#1}}}
    \newcommand{\RegionMarkerTok}[1]{{#1}}
    \newcommand{\ErrorTok}[1]{\textcolor[rgb]{1.00,0.00,0.00}{\textbf{{#1}}}}
    \newcommand{\NormalTok}[1]{{#1}}
    
    % Additional commands for more recent versions of Pandoc
    \newcommand{\ConstantTok}[1]{\textcolor[rgb]{0.53,0.00,0.00}{{#1}}}
    \newcommand{\SpecialCharTok}[1]{\textcolor[rgb]{0.25,0.44,0.63}{{#1}}}
    \newcommand{\VerbatimStringTok}[1]{\textcolor[rgb]{0.25,0.44,0.63}{{#1}}}
    \newcommand{\SpecialStringTok}[1]{\textcolor[rgb]{0.73,0.40,0.53}{{#1}}}
    \newcommand{\ImportTok}[1]{{#1}}
    \newcommand{\DocumentationTok}[1]{\textcolor[rgb]{0.73,0.13,0.13}{\textit{{#1}}}}
    \newcommand{\AnnotationTok}[1]{\textcolor[rgb]{0.38,0.63,0.69}{\textbf{\textit{{#1}}}}}
    \newcommand{\CommentVarTok}[1]{\textcolor[rgb]{0.38,0.63,0.69}{\textbf{\textit{{#1}}}}}
    \newcommand{\VariableTok}[1]{\textcolor[rgb]{0.10,0.09,0.49}{{#1}}}
    \newcommand{\ControlFlowTok}[1]{\textcolor[rgb]{0.00,0.44,0.13}{\textbf{{#1}}}}
    \newcommand{\OperatorTok}[1]{\textcolor[rgb]{0.40,0.40,0.40}{{#1}}}
    \newcommand{\BuiltInTok}[1]{{#1}}
    \newcommand{\ExtensionTok}[1]{{#1}}
    \newcommand{\PreprocessorTok}[1]{\textcolor[rgb]{0.74,0.48,0.00}{{#1}}}
    \newcommand{\AttributeTok}[1]{\textcolor[rgb]{0.49,0.56,0.16}{{#1}}}
    \newcommand{\InformationTok}[1]{\textcolor[rgb]{0.38,0.63,0.69}{\textbf{\textit{{#1}}}}}
    \newcommand{\WarningTok}[1]{\textcolor[rgb]{0.38,0.63,0.69}{\textbf{\textit{{#1}}}}}
    
    
    % Define a nice break command that doesn't care if a line doesn't already
    % exist.
    \def\br{\hspace*{\fill} \\* }
    % Math Jax compatibility definitions
    \def\gt{>}
    \def\lt{<}
    \let\Oldtex\TeX
    \let\Oldlatex\LaTeX
    \renewcommand{\TeX}{\textrm{\Oldtex}}
    \renewcommand{\LaTeX}{\textrm{\Oldlatex}}
    % Document parameters
    % Document title
    \title{A\_Adams\_aja149\_hw1}
    
    
    
    
    
% Pygments definitions
\makeatletter
\def\PY@reset{\let\PY@it=\relax \let\PY@bf=\relax%
    \let\PY@ul=\relax \let\PY@tc=\relax%
    \let\PY@bc=\relax \let\PY@ff=\relax}
\def\PY@tok#1{\csname PY@tok@#1\endcsname}
\def\PY@toks#1+{\ifx\relax#1\empty\else%
    \PY@tok{#1}\expandafter\PY@toks\fi}
\def\PY@do#1{\PY@bc{\PY@tc{\PY@ul{%
    \PY@it{\PY@bf{\PY@ff{#1}}}}}}}
\def\PY#1#2{\PY@reset\PY@toks#1+\relax+\PY@do{#2}}

\expandafter\def\csname PY@tok@w\endcsname{\def\PY@tc##1{\textcolor[rgb]{0.73,0.73,0.73}{##1}}}
\expandafter\def\csname PY@tok@c\endcsname{\let\PY@it=\textit\def\PY@tc##1{\textcolor[rgb]{0.25,0.50,0.50}{##1}}}
\expandafter\def\csname PY@tok@cp\endcsname{\def\PY@tc##1{\textcolor[rgb]{0.74,0.48,0.00}{##1}}}
\expandafter\def\csname PY@tok@k\endcsname{\let\PY@bf=\textbf\def\PY@tc##1{\textcolor[rgb]{0.00,0.50,0.00}{##1}}}
\expandafter\def\csname PY@tok@kp\endcsname{\def\PY@tc##1{\textcolor[rgb]{0.00,0.50,0.00}{##1}}}
\expandafter\def\csname PY@tok@kt\endcsname{\def\PY@tc##1{\textcolor[rgb]{0.69,0.00,0.25}{##1}}}
\expandafter\def\csname PY@tok@o\endcsname{\def\PY@tc##1{\textcolor[rgb]{0.40,0.40,0.40}{##1}}}
\expandafter\def\csname PY@tok@ow\endcsname{\let\PY@bf=\textbf\def\PY@tc##1{\textcolor[rgb]{0.67,0.13,1.00}{##1}}}
\expandafter\def\csname PY@tok@nb\endcsname{\def\PY@tc##1{\textcolor[rgb]{0.00,0.50,0.00}{##1}}}
\expandafter\def\csname PY@tok@nf\endcsname{\def\PY@tc##1{\textcolor[rgb]{0.00,0.00,1.00}{##1}}}
\expandafter\def\csname PY@tok@nc\endcsname{\let\PY@bf=\textbf\def\PY@tc##1{\textcolor[rgb]{0.00,0.00,1.00}{##1}}}
\expandafter\def\csname PY@tok@nn\endcsname{\let\PY@bf=\textbf\def\PY@tc##1{\textcolor[rgb]{0.00,0.00,1.00}{##1}}}
\expandafter\def\csname PY@tok@ne\endcsname{\let\PY@bf=\textbf\def\PY@tc##1{\textcolor[rgb]{0.82,0.25,0.23}{##1}}}
\expandafter\def\csname PY@tok@nv\endcsname{\def\PY@tc##1{\textcolor[rgb]{0.10,0.09,0.49}{##1}}}
\expandafter\def\csname PY@tok@no\endcsname{\def\PY@tc##1{\textcolor[rgb]{0.53,0.00,0.00}{##1}}}
\expandafter\def\csname PY@tok@nl\endcsname{\def\PY@tc##1{\textcolor[rgb]{0.63,0.63,0.00}{##1}}}
\expandafter\def\csname PY@tok@ni\endcsname{\let\PY@bf=\textbf\def\PY@tc##1{\textcolor[rgb]{0.60,0.60,0.60}{##1}}}
\expandafter\def\csname PY@tok@na\endcsname{\def\PY@tc##1{\textcolor[rgb]{0.49,0.56,0.16}{##1}}}
\expandafter\def\csname PY@tok@nt\endcsname{\let\PY@bf=\textbf\def\PY@tc##1{\textcolor[rgb]{0.00,0.50,0.00}{##1}}}
\expandafter\def\csname PY@tok@nd\endcsname{\def\PY@tc##1{\textcolor[rgb]{0.67,0.13,1.00}{##1}}}
\expandafter\def\csname PY@tok@s\endcsname{\def\PY@tc##1{\textcolor[rgb]{0.73,0.13,0.13}{##1}}}
\expandafter\def\csname PY@tok@sd\endcsname{\let\PY@it=\textit\def\PY@tc##1{\textcolor[rgb]{0.73,0.13,0.13}{##1}}}
\expandafter\def\csname PY@tok@si\endcsname{\let\PY@bf=\textbf\def\PY@tc##1{\textcolor[rgb]{0.73,0.40,0.53}{##1}}}
\expandafter\def\csname PY@tok@se\endcsname{\let\PY@bf=\textbf\def\PY@tc##1{\textcolor[rgb]{0.73,0.40,0.13}{##1}}}
\expandafter\def\csname PY@tok@sr\endcsname{\def\PY@tc##1{\textcolor[rgb]{0.73,0.40,0.53}{##1}}}
\expandafter\def\csname PY@tok@ss\endcsname{\def\PY@tc##1{\textcolor[rgb]{0.10,0.09,0.49}{##1}}}
\expandafter\def\csname PY@tok@sx\endcsname{\def\PY@tc##1{\textcolor[rgb]{0.00,0.50,0.00}{##1}}}
\expandafter\def\csname PY@tok@m\endcsname{\def\PY@tc##1{\textcolor[rgb]{0.40,0.40,0.40}{##1}}}
\expandafter\def\csname PY@tok@gh\endcsname{\let\PY@bf=\textbf\def\PY@tc##1{\textcolor[rgb]{0.00,0.00,0.50}{##1}}}
\expandafter\def\csname PY@tok@gu\endcsname{\let\PY@bf=\textbf\def\PY@tc##1{\textcolor[rgb]{0.50,0.00,0.50}{##1}}}
\expandafter\def\csname PY@tok@gd\endcsname{\def\PY@tc##1{\textcolor[rgb]{0.63,0.00,0.00}{##1}}}
\expandafter\def\csname PY@tok@gi\endcsname{\def\PY@tc##1{\textcolor[rgb]{0.00,0.63,0.00}{##1}}}
\expandafter\def\csname PY@tok@gr\endcsname{\def\PY@tc##1{\textcolor[rgb]{1.00,0.00,0.00}{##1}}}
\expandafter\def\csname PY@tok@ge\endcsname{\let\PY@it=\textit}
\expandafter\def\csname PY@tok@gs\endcsname{\let\PY@bf=\textbf}
\expandafter\def\csname PY@tok@gp\endcsname{\let\PY@bf=\textbf\def\PY@tc##1{\textcolor[rgb]{0.00,0.00,0.50}{##1}}}
\expandafter\def\csname PY@tok@go\endcsname{\def\PY@tc##1{\textcolor[rgb]{0.53,0.53,0.53}{##1}}}
\expandafter\def\csname PY@tok@gt\endcsname{\def\PY@tc##1{\textcolor[rgb]{0.00,0.27,0.87}{##1}}}
\expandafter\def\csname PY@tok@err\endcsname{\def\PY@bc##1{\setlength{\fboxsep}{0pt}\fcolorbox[rgb]{1.00,0.00,0.00}{1,1,1}{\strut ##1}}}
\expandafter\def\csname PY@tok@kc\endcsname{\let\PY@bf=\textbf\def\PY@tc##1{\textcolor[rgb]{0.00,0.50,0.00}{##1}}}
\expandafter\def\csname PY@tok@kd\endcsname{\let\PY@bf=\textbf\def\PY@tc##1{\textcolor[rgb]{0.00,0.50,0.00}{##1}}}
\expandafter\def\csname PY@tok@kn\endcsname{\let\PY@bf=\textbf\def\PY@tc##1{\textcolor[rgb]{0.00,0.50,0.00}{##1}}}
\expandafter\def\csname PY@tok@kr\endcsname{\let\PY@bf=\textbf\def\PY@tc##1{\textcolor[rgb]{0.00,0.50,0.00}{##1}}}
\expandafter\def\csname PY@tok@bp\endcsname{\def\PY@tc##1{\textcolor[rgb]{0.00,0.50,0.00}{##1}}}
\expandafter\def\csname PY@tok@fm\endcsname{\def\PY@tc##1{\textcolor[rgb]{0.00,0.00,1.00}{##1}}}
\expandafter\def\csname PY@tok@vc\endcsname{\def\PY@tc##1{\textcolor[rgb]{0.10,0.09,0.49}{##1}}}
\expandafter\def\csname PY@tok@vg\endcsname{\def\PY@tc##1{\textcolor[rgb]{0.10,0.09,0.49}{##1}}}
\expandafter\def\csname PY@tok@vi\endcsname{\def\PY@tc##1{\textcolor[rgb]{0.10,0.09,0.49}{##1}}}
\expandafter\def\csname PY@tok@vm\endcsname{\def\PY@tc##1{\textcolor[rgb]{0.10,0.09,0.49}{##1}}}
\expandafter\def\csname PY@tok@sa\endcsname{\def\PY@tc##1{\textcolor[rgb]{0.73,0.13,0.13}{##1}}}
\expandafter\def\csname PY@tok@sb\endcsname{\def\PY@tc##1{\textcolor[rgb]{0.73,0.13,0.13}{##1}}}
\expandafter\def\csname PY@tok@sc\endcsname{\def\PY@tc##1{\textcolor[rgb]{0.73,0.13,0.13}{##1}}}
\expandafter\def\csname PY@tok@dl\endcsname{\def\PY@tc##1{\textcolor[rgb]{0.73,0.13,0.13}{##1}}}
\expandafter\def\csname PY@tok@s2\endcsname{\def\PY@tc##1{\textcolor[rgb]{0.73,0.13,0.13}{##1}}}
\expandafter\def\csname PY@tok@sh\endcsname{\def\PY@tc##1{\textcolor[rgb]{0.73,0.13,0.13}{##1}}}
\expandafter\def\csname PY@tok@s1\endcsname{\def\PY@tc##1{\textcolor[rgb]{0.73,0.13,0.13}{##1}}}
\expandafter\def\csname PY@tok@mb\endcsname{\def\PY@tc##1{\textcolor[rgb]{0.40,0.40,0.40}{##1}}}
\expandafter\def\csname PY@tok@mf\endcsname{\def\PY@tc##1{\textcolor[rgb]{0.40,0.40,0.40}{##1}}}
\expandafter\def\csname PY@tok@mh\endcsname{\def\PY@tc##1{\textcolor[rgb]{0.40,0.40,0.40}{##1}}}
\expandafter\def\csname PY@tok@mi\endcsname{\def\PY@tc##1{\textcolor[rgb]{0.40,0.40,0.40}{##1}}}
\expandafter\def\csname PY@tok@il\endcsname{\def\PY@tc##1{\textcolor[rgb]{0.40,0.40,0.40}{##1}}}
\expandafter\def\csname PY@tok@mo\endcsname{\def\PY@tc##1{\textcolor[rgb]{0.40,0.40,0.40}{##1}}}
\expandafter\def\csname PY@tok@ch\endcsname{\let\PY@it=\textit\def\PY@tc##1{\textcolor[rgb]{0.25,0.50,0.50}{##1}}}
\expandafter\def\csname PY@tok@cm\endcsname{\let\PY@it=\textit\def\PY@tc##1{\textcolor[rgb]{0.25,0.50,0.50}{##1}}}
\expandafter\def\csname PY@tok@cpf\endcsname{\let\PY@it=\textit\def\PY@tc##1{\textcolor[rgb]{0.25,0.50,0.50}{##1}}}
\expandafter\def\csname PY@tok@c1\endcsname{\let\PY@it=\textit\def\PY@tc##1{\textcolor[rgb]{0.25,0.50,0.50}{##1}}}
\expandafter\def\csname PY@tok@cs\endcsname{\let\PY@it=\textit\def\PY@tc##1{\textcolor[rgb]{0.25,0.50,0.50}{##1}}}

\def\PYZbs{\char`\\}
\def\PYZus{\char`\_}
\def\PYZob{\char`\{}
\def\PYZcb{\char`\}}
\def\PYZca{\char`\^}
\def\PYZam{\char`\&}
\def\PYZlt{\char`\<}
\def\PYZgt{\char`\>}
\def\PYZsh{\char`\#}
\def\PYZpc{\char`\%}
\def\PYZdl{\char`\$}
\def\PYZhy{\char`\-}
\def\PYZsq{\char`\'}
\def\PYZdq{\char`\"}
\def\PYZti{\char`\~}
% for compatibility with earlier versions
\def\PYZat{@}
\def\PYZlb{[}
\def\PYZrb{]}
\makeatother


    % For linebreaks inside Verbatim environment from package fancyvrb. 
    \makeatletter
        \newbox\Wrappedcontinuationbox 
        \newbox\Wrappedvisiblespacebox 
        \newcommand*\Wrappedvisiblespace {\textcolor{red}{\textvisiblespace}} 
        \newcommand*\Wrappedcontinuationsymbol {\textcolor{red}{\llap{\tiny$\m@th\hookrightarrow$}}} 
        \newcommand*\Wrappedcontinuationindent {3ex } 
        \newcommand*\Wrappedafterbreak {\kern\Wrappedcontinuationindent\copy\Wrappedcontinuationbox} 
        % Take advantage of the already applied Pygments mark-up to insert 
        % potential linebreaks for TeX processing. 
        %        {, <, #, %, $, ' and ": go to next line. 
        %        _, }, ^, &, >, - and ~: stay at end of broken line. 
        % Use of \textquotesingle for straight quote. 
        \newcommand*\Wrappedbreaksatspecials {% 
            \def\PYGZus{\discretionary{\char`\_}{\Wrappedafterbreak}{\char`\_}}% 
            \def\PYGZob{\discretionary{}{\Wrappedafterbreak\char`\{}{\char`\{}}% 
            \def\PYGZcb{\discretionary{\char`\}}{\Wrappedafterbreak}{\char`\}}}% 
            \def\PYGZca{\discretionary{\char`\^}{\Wrappedafterbreak}{\char`\^}}% 
            \def\PYGZam{\discretionary{\char`\&}{\Wrappedafterbreak}{\char`\&}}% 
            \def\PYGZlt{\discretionary{}{\Wrappedafterbreak\char`\<}{\char`\<}}% 
            \def\PYGZgt{\discretionary{\char`\>}{\Wrappedafterbreak}{\char`\>}}% 
            \def\PYGZsh{\discretionary{}{\Wrappedafterbreak\char`\#}{\char`\#}}% 
            \def\PYGZpc{\discretionary{}{\Wrappedafterbreak\char`\%}{\char`\%}}% 
            \def\PYGZdl{\discretionary{}{\Wrappedafterbreak\char`\$}{\char`\$}}% 
            \def\PYGZhy{\discretionary{\char`\-}{\Wrappedafterbreak}{\char`\-}}% 
            \def\PYGZsq{\discretionary{}{\Wrappedafterbreak\textquotesingle}{\textquotesingle}}% 
            \def\PYGZdq{\discretionary{}{\Wrappedafterbreak\char`\"}{\char`\"}}% 
            \def\PYGZti{\discretionary{\char`\~}{\Wrappedafterbreak}{\char`\~}}% 
        } 
        % Some characters . , ; ? ! / are not pygmentized. 
        % This macro makes them "active" and they will insert potential linebreaks 
        \newcommand*\Wrappedbreaksatpunct {% 
            \lccode`\~`\.\lowercase{\def~}{\discretionary{\hbox{\char`\.}}{\Wrappedafterbreak}{\hbox{\char`\.}}}% 
            \lccode`\~`\,\lowercase{\def~}{\discretionary{\hbox{\char`\,}}{\Wrappedafterbreak}{\hbox{\char`\,}}}% 
            \lccode`\~`\;\lowercase{\def~}{\discretionary{\hbox{\char`\;}}{\Wrappedafterbreak}{\hbox{\char`\;}}}% 
            \lccode`\~`\:\lowercase{\def~}{\discretionary{\hbox{\char`\:}}{\Wrappedafterbreak}{\hbox{\char`\:}}}% 
            \lccode`\~`\?\lowercase{\def~}{\discretionary{\hbox{\char`\?}}{\Wrappedafterbreak}{\hbox{\char`\?}}}% 
            \lccode`\~`\!\lowercase{\def~}{\discretionary{\hbox{\char`\!}}{\Wrappedafterbreak}{\hbox{\char`\!}}}% 
            \lccode`\~`\/\lowercase{\def~}{\discretionary{\hbox{\char`\/}}{\Wrappedafterbreak}{\hbox{\char`\/}}}% 
            \catcode`\.\active
            \catcode`\,\active 
            \catcode`\;\active
            \catcode`\:\active
            \catcode`\?\active
            \catcode`\!\active
            \catcode`\/\active 
            \lccode`\~`\~ 	
        }
    \makeatother

    \let\OriginalVerbatim=\Verbatim
    \makeatletter
    \renewcommand{\Verbatim}[1][1]{%
        %\parskip\z@skip
        \sbox\Wrappedcontinuationbox {\Wrappedcontinuationsymbol}%
        \sbox\Wrappedvisiblespacebox {\FV@SetupFont\Wrappedvisiblespace}%
        \def\FancyVerbFormatLine ##1{\hsize\linewidth
            \vtop{\raggedright\hyphenpenalty\z@\exhyphenpenalty\z@
                \doublehyphendemerits\z@\finalhyphendemerits\z@
                \strut ##1\strut}%
        }%
        % If the linebreak is at a space, the latter will be displayed as visible
        % space at end of first line, and a continuation symbol starts next line.
        % Stretch/shrink are however usually zero for typewriter font.
        \def\FV@Space {%
            \nobreak\hskip\z@ plus\fontdimen3\font minus\fontdimen4\font
            \discretionary{\copy\Wrappedvisiblespacebox}{\Wrappedafterbreak}
            {\kern\fontdimen2\font}%
        }%
        
        % Allow breaks at special characters using \PYG... macros.
        \Wrappedbreaksatspecials
        % Breaks at punctuation characters . , ; ? ! and / need catcode=\active 	
        \OriginalVerbatim[#1,codes*=\Wrappedbreaksatpunct]%
    }
    \makeatother

    % Exact colors from NB
    \definecolor{incolor}{HTML}{303F9F}
    \definecolor{outcolor}{HTML}{D84315}
    \definecolor{cellborder}{HTML}{CFCFCF}
    \definecolor{cellbackground}{HTML}{F7F7F7}
    
    % prompt
    \makeatletter
    \newcommand{\boxspacing}{\kern\kvtcb@left@rule\kern\kvtcb@boxsep}
    \makeatother
    \newcommand{\prompt}[4]{
        \ttfamily\llap{{\color{#2}[#3]:\hspace{3pt}#4}}\vspace{-\baselineskip}
    }
    

    
    % Prevent overflowing lines due to hard-to-break entities
    \sloppy 
    % Setup hyperref package
    \hypersetup{
      breaklinks=true,  % so long urls are correctly broken across lines
      colorlinks=true,
      urlcolor=urlcolor,
      linkcolor=linkcolor,
      citecolor=citecolor,
      }
    % Slightly bigger margins than the latex defaults
    
    \geometry{verbose,tmargin=1in,bmargin=1in,lmargin=1in,rmargin=1in}
    
    

\begin{document}
    
    \maketitle
    
    

    
    \hypertarget{alexander-adams}{%
\section{Alexander Adams}\label{alexander-adams}}

\hypertarget{aja149}{%
\section{aja149}\label{aja149}}

\hypertarget{homework-1-words-words-words}{%
\section{Homework \#1: Words, Words,
Words}\label{homework-1-words-words-words}}

\hypertarget{ppol628-text-as-data}{%
\section{PPOL628 Text as Data}\label{ppol628-text-as-data}}

    \hypertarget{homework-state-your-assumptions}{%
\section{Homework: State Your
Assumptions}\label{homework-state-your-assumptions}}

\begin{quote}
POLONIUS What do you read, my lord?
\end{quote}

\begin{quote}
HAMLET Words, words, words
\end{quote}

\begin{quote}
-- Hamlet, Act 2, Scene 2
\end{quote}

\hypertarget{this-homework-deals-with-the-assumptions-made-when-taking-text-from-its-original-raw-form-into-something-more-computable.}{%
\subsection{This homework deals with the assumptions made when taking
text from its original ``raw'' form into something more
computable.}\label{this-homework-deals-with-the-assumptions-made-when-taking-text-from-its-original-raw-form-into-something-more-computable.}}

\begin{itemize}
\tightlist
\item
  Assumptions about the shape of text (e.g.~how to break a corpus into
  documents)
\item
  Assumptions about what makes a token, an entity, etc.
\item
  Assumptions about what interesting or important content looks like,
  and how that informs our analyses.
\end{itemize}

\hypertarget{there-are-three-parts}{%
\subsection{There are three parts:}\label{there-are-three-parts}}

\begin{enumerate}
\def\labelenumi{\arabic{enumi}.}
\tightlist
\item
  Splitting Lines from Shakespeare
\item
  Tokenizing and Aligning lines into plays
\item
  Assessing and comparing characters from within each play
\end{enumerate}

\emph{NB This file is merely a template, with instructions; do not feel
constrained to using it directly if you do not wish to.}

\begin{center}\rule{0.5\linewidth}{0.5pt}\end{center}

    \hypertarget{get-the-data}{%
\subsection{Get the Data}\label{get-the-data}}

Since the class uses dvc, it is possible to get this dataset either
using the command line (e.g.~dvc import
https://github.com/TLP-COI/text-data-course
resources/data/shakespeare/shakespeare.txt), or using the python api (if
you wish to use python):

    \begin{tcolorbox}[breakable, size=fbox, boxrule=1pt, pad at break*=1mm,colback=cellbackground, colframe=cellborder]
\prompt{In}{incolor}{1}{\boxspacing}
\begin{Verbatim}[commandchars=\\\{\}]
\PY{c+c1}{\PYZsh{}I could not get the dvc.api read function to work. It consistenly produced the following error:}
\PY{c+c1}{\PYZsh{}AttributeError: \PYZsq{}HashFile\PYZsq{} object has no attribute \PYZsq{}get\PYZsq{}}

\PY{c+c1}{\PYZsh{}I switched to this import method because it was the only way I could actually read in the text file.}

\PY{c+c1}{\PYZsh{}from dvc.api import read,get\PYZus{}url}
\PY{c+c1}{\PYZsh{}import pandas as pd}

\PY{c+c1}{\PYZsh{}txt = pd.read\PYZus{}csv(\PYZsq{}shakespeare.txt\PYZsq{}, sep = \PYZsq{}\PYZbs{}n\PYZsq{}, header = None)}

\PY{c+c1}{\PYZsh{}print(txt[:250])}
\end{Verbatim}
\end{tcolorbox}

    \begin{tcolorbox}[breakable, size=fbox, boxrule=1pt, pad at break*=1mm,colback=cellbackground, colframe=cellborder]
\prompt{In}{incolor}{2}{\boxspacing}
\begin{Verbatim}[commandchars=\\\{\}]
\PY{c+c1}{\PYZsh{}Use relative path because shakespeare.txt is in parent directory}
\PY{k}{with} \PY{n+nb}{open}\PY{p}{(}\PY{l+s+s1}{\PYZsq{}}\PY{l+s+s1}{..}\PY{l+s+s1}{\PYZbs{}}\PY{l+s+s1}{..}\PY{l+s+s1}{\PYZbs{}}\PY{l+s+s1}{..}\PY{l+s+s1}{\PYZbs{}}\PY{l+s+s1}{shakespeare.txt}\PY{l+s+s1}{\PYZsq{}}\PY{p}{)} \PY{k}{as} \PY{n}{f}\PY{p}{:}
    \PY{n}{lines} \PY{o}{=} \PY{n}{f}\PY{o}{.}\PY{n}{read}\PY{p}{(}\PY{p}{)}
\end{Verbatim}
\end{tcolorbox}

    Make sure this works before you continue! Either way, it would likely be
beneficial to have the data downloaded locally to keep from needing to
re-dowload it every time.

\begin{center}\rule{0.5\linewidth}{0.5pt}\end{center}

    \hypertarget{part-1}{%
\section{Part 1}\label{part-1}}

Split the text file into a table, such that:

\begin{verbatim}
-each row is a single line of dialogue
-there are columns for:
    *the speaker
    *the line number
    *the line dialogue (the text)
\end{verbatim}

\emph{Hint: you will need to use RegEx to do this rapidly. See the
in-class ``markdown'' example!}

Question(s):

\begin{verbatim}
*What assumptions have you made about the text that allowed you to do this?
\end{verbatim}

\begin{center}\rule{0.5\linewidth}{0.5pt}\end{center}

    \begin{tcolorbox}[breakable, size=fbox, boxrule=1pt, pad at break*=1mm,colback=cellbackground, colframe=cellborder]
\prompt{In}{incolor}{3}{\boxspacing}
\begin{Verbatim}[commandchars=\\\{\}]
\PY{c+c1}{\PYZsh{}Import the re package for regex}
\PY{k+kn}{import} \PY{n+nn}{pandas} \PY{k}{as} \PY{n+nn}{pd}
\PY{n}{pd}\PY{o}{.}\PY{n}{set\PYZus{}option}\PY{p}{(}\PY{l+s+s1}{\PYZsq{}}\PY{l+s+s1}{display.max\PYZus{}rows}\PY{l+s+s1}{\PYZsq{}}\PY{p}{,} \PY{k+kc}{None}\PY{p}{)}
\PY{n}{pd}\PY{o}{.}\PY{n}{set\PYZus{}option}\PY{p}{(}\PY{l+s+s1}{\PYZsq{}}\PY{l+s+s1}{display.width}\PY{l+s+s1}{\PYZsq{}}\PY{p}{,} \PY{k+kc}{None}\PY{p}{)}
\PY{n}{pd}\PY{o}{.}\PY{n}{set\PYZus{}option}\PY{p}{(}\PY{l+s+s1}{\PYZsq{}}\PY{l+s+s1}{display.max\PYZus{}colwidth}\PY{l+s+s1}{\PYZsq{}}\PY{p}{,} \PY{k+kc}{None}\PY{p}{)}
\PY{k+kn}{import} \PY{n+nn}{re}
\end{Verbatim}
\end{tcolorbox}

    Each chunk of text contains two parts, a speaker and the dialogue.

The key assumptions to be made here are that the speaker's name is
always followed by a colon and a new line(``:\n''), and each chunk of
text is followed by two new lines:

Speaker\_Name:\n Text\n\n

    Note: In the interest of full disclosure, I initially tried to get away
with using as little regex for this part as possible. I've preserved my
original solution (which works perfectly fine as far as I can tell, but
eschews regex almost entirely in favor of pandas string methods) as
commented code below.

The regex patterns below were posted by Professor Sexton and Ella Zhang
in our class slack. I've included original explanations, and tried to
modify the patterns so that I wasn't just copying outright. (The only
real modification I made was swapping out a plus sign in the first
pattern for \{1,\}. While this is functionally the same but longer and
therefore worse, it's at least an original contribution to this.)

    \begin{tcolorbox}[breakable, size=fbox, boxrule=1pt, pad at break*=1mm,colback=cellbackground, colframe=cellborder]
\prompt{In}{incolor}{103}{\boxspacing}
\begin{Verbatim}[commandchars=\\\{\}]
\PY{c+c1}{\PYZsh{}Goal: create two patterns}
\PY{c+c1}{\PYZsh{}1) Pattern to capture speaker}
\PY{c+c1}{\PYZsh{}2) Pattern to capture dialogue}

\PY{n}{shakespeare\PYZus{}pattern} \PY{o}{=} \PY{n}{re}\PY{o}{.}\PY{n}{compile}\PY{p}{(}
    \PY{l+s+s2}{\PYZdq{}}\PY{l+s+s2}{(\PYZca{}[A\PYZhy{}Z].}\PY{l+s+s2}{\PYZob{}}\PY{l+s+s2}{1,\PYZcb{}?):\PYZdl{}}\PY{l+s+s2}{\PYZdq{}} \PY{c+c1}{\PYZsh{}Speaker pattern}
    \PY{l+s+s2}{\PYZdq{}}\PY{l+s+se}{\PYZbs{}n}\PY{l+s+si}{\PYZob{}1\PYZcb{}}\PY{l+s+s2}{(.*?)}\PY{l+s+se}{\PYZbs{}n}\PY{l+s+si}{\PYZob{}2\PYZcb{}}\PY{l+s+s2}{\PYZdq{}}\PY{p}{,} \PY{c+c1}{\PYZsh{}Dialogue pattern}
    \PY{n}{flags} \PY{o}{=} \PY{n}{re}\PY{o}{.}\PY{n}{S} \PY{o}{|} \PY{n}{re}\PY{o}{.}\PY{n}{M}
\PY{p}{)}
\end{Verbatim}
\end{tcolorbox}

    The first pattern above is as follows: (\footnote{A-Z}.+?):\$. First,
let's consider the characters inside the parentheses. The caret
indicates the start of a pattern. The square brackets with A-Z in them
mean that there will be a capital letter. Periods can be any character,
and a character followed by a number in curly braces must occur that
many times. In this case, the number is followed by a comma, which means
the preceding token can occur that number of times or more. The question
mark modifies the number-comma in curly braces by making it ``lazy''. In
regex terms, this means that while it can match any number of characters
corresponding to the preceding token, a lazy version will match the
minimum characters necessary. Taken all together, this pattern means
that the line must start with a capital letter, then be followed by any
sequence of characters, as long as it takes but as few as possible to
fulfill the pattern. The colon is standard character (in this case, it's
intended to match the colon after the speaker's name), and the dollar
sign ends the line.

The second pattern above is as follows: \n{1}(.*?)\n{2}. Conceptually,
this one is a bit simpler. ``\n'' indicates a new line, and a number in
curly brackets hard-codes the number of times the preceding token can
occur. Thus, \n{1} will match an occurrence of one new line. The
parentheses demarcate a group, as before. The period can be anything,
and the asterisk means that token can occur any number of times. The
question mark, as before, makes the asterisk lazy. All together, this
parenthetical group matches any token, any number of times, as few
tokens as possible. This pattern is then followed by \n{2}, which
matches two new lines. All together, this pattern captures the new line
at the start of the line, then all the textual characters on the row,
and then the two lines which separate this chunk of text from the next
one.

The last parts of the regex expression above are the two flags. In the
\texttt{re} package for Python, \texttt{re.S} allows the period to match
a new line character. Without this flag, the period can match any
character except a new line. The \texttt{re.M} flag lets the caret and
dollar sign work at the start and end of a given line, rather than just
at the start and end of a whole string. These flags are separated by an
OR operator, indicating that only one needs to match for a match to be
valid.

    \begin{tcolorbox}[breakable, size=fbox, boxrule=1pt, pad at break*=1mm,colback=cellbackground, colframe=cellborder]
\prompt{In}{incolor}{106}{\boxspacing}
\begin{Verbatim}[commandchars=\\\{\}]
\PY{n}{text} \PY{o}{=} \PY{n}{shakespeare\PYZus{}pattern}\PY{o}{.}\PY{n}{findall}\PY{p}{(}\PY{n}{lines}\PY{p}{)}
\end{Verbatim}
\end{tcolorbox}

    \begin{tcolorbox}[breakable, size=fbox, boxrule=1pt, pad at break*=1mm,colback=cellbackground, colframe=cellborder]
\prompt{In}{incolor}{109}{\boxspacing}
\begin{Verbatim}[commandchars=\\\{\}]
\PY{n}{text} \PY{o}{=} \PY{n}{pd}\PY{o}{.}\PY{n}{DataFrame}\PY{p}{(}\PY{n}{matches1}\PY{p}{)}
\end{Verbatim}
\end{tcolorbox}

    \begin{tcolorbox}[breakable, size=fbox, boxrule=1pt, pad at break*=1mm,colback=cellbackground, colframe=cellborder]
\prompt{In}{incolor}{110}{\boxspacing}
\begin{Verbatim}[commandchars=\\\{\}]
\PY{c+c1}{\PYZsh{}Rename columns for ease of use}
\PY{n}{text} \PY{o}{=} \PY{n}{text}\PY{o}{.}\PY{n}{rename}\PY{p}{(}\PY{n}{columns} \PY{o}{=} \PY{p}{\PYZob{}}\PY{l+m+mi}{0}\PY{p}{:} \PY{l+s+s1}{\PYZsq{}}\PY{l+s+s1}{Speaker}\PY{l+s+s1}{\PYZsq{}}\PY{p}{,}
                              \PY{l+m+mi}{1}\PY{p}{:} \PY{l+s+s1}{\PYZsq{}}\PY{l+s+s1}{Text}\PY{l+s+s1}{\PYZsq{}}\PY{p}{\PYZcb{}}\PY{p}{)}
\end{Verbatim}
\end{tcolorbox}

    \begin{tcolorbox}[breakable, size=fbox, boxrule=1pt, pad at break*=1mm,colback=cellbackground, colframe=cellborder]
\prompt{In}{incolor}{4}{\boxspacing}
\begin{Verbatim}[commandchars=\\\{\}]
\PY{c+c1}{\PYZsh{}Original solution with minimal regex:}

\PY{c+c1}{\PYZsh{}1) Split the text using two new lines as a delimiter}
\PY{c+c1}{\PYZsh{}text = re.split(r\PYZsq{}\PYZbs{}n\PYZbs{}n\PYZsq{},lines)}

\PY{c+c1}{\PYZsh{}2) Convert to a pandas dataframe}
\PY{c+c1}{\PYZsh{}text = pd.DataFrame(text, columns = [\PYZsq{}text\PYZsq{}])}

\PY{c+c1}{\PYZsh{}3) Call the split method on the string attribute to split the text column into 2, }
\PY{c+c1}{\PYZsh{}using the colon followed by a new line as the delimiter. }
\PY{c+c1}{\PYZsh{}text = text[\PYZsq{}text\PYZsq{}].str.split(\PYZsq{}:\PYZbs{}n\PYZsq{}, 1, expand=True)}

\PY{c+c1}{\PYZsh{}\PYZti{}\PYZti{}4) Then use the `split` method again to break each chunk up into individual lines, followed by }
\PY{c+c1}{\PYZsh{}the `explode` function to make each individual line its own row. Call `reset\PYZus{}index` with }
\PY{c+c1}{\PYZsh{}`drop = True` to drop the index column, which has the same value for all lines which were part}
\PY{c+c1}{\PYZsh{}of the same original chunk.\PYZti{}\PYZti{}}

\PY{c+c1}{\PYZsh{}EDIT: As per the discussion in the slack channel, this step is not necessary. I\PYZsq{}ve included }
\PY{c+c1}{\PYZsh{}the below two cells, commented out, to show what I would have done to split each chunk of }
\PY{c+c1}{\PYZsh{}dialogue into individual lines. }

\PY{c+c1}{\PYZsh{}text[1] = text[1].str.split(\PYZsq{}\PYZbs{}n\PYZsq{})}
\PY{c+c1}{\PYZsh{}text = text.explode(1).reset\PYZus{}index(drop = True)}


\PY{c+c1}{\PYZsh{}5) Finally, reset the index to create a column of integer values for line numbers, }
\PY{c+c1}{\PYZsh{}and then add one to each item in that column because Python is a zero\PYZhy{}indexed language. }
\PY{c+c1}{\PYZsh{}Rename the columns in the data frame for clarity. }
\PY{c+c1}{\PYZsh{}text = text.reset\PYZus{}index().rename(columns = \PYZob{}\PYZsq{}index\PYZsq{}: \PYZsq{}Line\PYZus{}Number\PYZsq{}, }
\PY{c+c1}{\PYZsh{}                                            0: \PYZsq{}Speaker\PYZsq{},}
\PY{c+c1}{\PYZsh{}                                            1: \PYZsq{}Text\PYZsq{}\PYZcb{})}
\PY{c+c1}{\PYZsh{}text[\PYZsq{}Line\PYZus{}Number\PYZsq{}] = text[\PYZsq{}Line\PYZus{}Number\PYZsq{}]+1}
\PY{c+c1}{\PYZsh{}text.head(10)}
\end{Verbatim}
\end{tcolorbox}

    \begin{longtable}[]{@{}l@{}}
\toprule
\endhead
\begin{minipage}[t]{0.07\columnwidth}\raggedright
\#\# Part 2\strut
\end{minipage}\tabularnewline
\begin{minipage}[t]{0.07\columnwidth}\raggedright
\#\#\#\# You have likely noticed that the lines are not all from the
same play! Now, we will add some useful metadata to our table:\strut
\end{minipage}\tabularnewline
\begin{minipage}[t]{0.07\columnwidth}\raggedright
-Determine a likely source title for each line. -Add the title as a
`play' column in the data table. -Make sure to document your decisions,
assumptions, external data sources, etc.\strut
\end{minipage}\tabularnewline
\begin{minipage}[t]{0.07\columnwidth}\raggedright
This is fairly open-ended, and you are not being judged completely on
accuracy. Instead, think outside the box a bit as to how you might
accomplish this, and attempt to justify whatever approximations or
assumptions you felt were appropriate.\strut
\end{minipage}\tabularnewline
\bottomrule
\end{longtable}

    \begin{tcolorbox}[breakable, size=fbox, boxrule=1pt, pad at break*=1mm,colback=cellbackground, colframe=cellborder]
\prompt{In}{incolor}{111}{\boxspacing}
\begin{Verbatim}[commandchars=\\\{\}]
\PY{c+c1}{\PYZsh{}Just for fun, let\PYZsq{}s see a list of all unique speakers in the table:}
\PY{n}{text}\PY{o}{.}\PY{n}{Speaker}\PY{o}{.}\PY{n}{unique}\PY{p}{(}\PY{p}{)}
\end{Verbatim}
\end{tcolorbox}

            \begin{tcolorbox}[breakable, size=fbox, boxrule=.5pt, pad at break*=1mm, opacityfill=0]
\prompt{Out}{outcolor}{111}{\boxspacing}
\begin{Verbatim}[commandchars=\\\{\}]
array(['First Citizen', 'All', 'Second Citizen', 'MENENIUS', 'MARCIUS',
       'Messenger', 'First Senator', 'COMINIUS', 'TITUS', 'SICINIUS',
       'BRUTUS', 'AUFIDIUS', 'Second Senator', 'VOLUMNIA', 'VIRGILIA',
       'Gentlewoman', 'VALERIA', 'LARTIUS', 'First Soldier',
       'Second Soldier', 'First Roman', 'Second Roman', 'Third Roman',
       'Lieutenant', 'CORIOLANUS', 'Both', 'Herald', 'First Officer',
       'Second Officer', 'Officer', 'Senators', 'Third Citizen',
       'Fourth Citizen', 'Fifth Citizen', 'Both Citizens',
       'Sixth Citizen', 'Seventh Citizen', 'All Citizens', 'Citizens',
       'Senators, \&C', 'AEdile', 'A Patrician', 'Second Patrician',
       'Both Tribunes', 'Roman', 'Volsce', 'Citizen', 'First Servingman',
       'Second Servingman', 'Third Servingman', 'Second Messenger',
       'Young MARCIUS', 'First Conspirator', 'Second Conspirator',
       'Third Conspirator', 'All The Lords', 'Lords', 'First Lord',
       'All Conspirators', 'All The People', 'Second Lord', 'Third Lord',
       'GLOUCESTER', 'CLARENCE', 'BRAKENBURY', 'HASTINGS', 'LADY ANNE',
       'Gentleman', 'GENTLEMEN', 'RIVERS', 'GREY', 'QUEEN ELIZABETH',
       'BUCKINGHAM', 'DERBY', 'QUEEN MARGARET', 'DORSET', 'CATESBY',
       'First Murderer', 'Second Murderer', 'Second murderer',
       'KING EDWARD IV', 'Boy', 'DUCHESS OF YORK', 'Girl', 'Children',
       'ARCHBISHOP OF YORK', 'YORK', 'PRINCE EDWARD', 'Lord Mayor',
       'CARDINAL', 'STANLEY', 'LORD STANLEY', 'Pursuivant', 'Priest',
       'RATCLIFF', 'VAUGHAN', 'BISHOP OF ELY', 'LOVEL', 'Scrivener',
       'ANOTHER', 'KING RICHARD III', 'Page', 'TYRREL', 'Third Messenger',
       'Fourth Messenger', 'CHRISTOPHER', 'Sheriff', 'RICHMOND', 'OXFORD',
       'HERBERT', 'BLUNT', 'SURREY', 'NORFOLK', 'Ghost of Prince Edward',
       'Ghost of CLARENCE', 'Ghost of GREY', 'Ghosts of young Princes',
       'Ghost of BUCKINGHAM', 'LORDS', 'KING RICHARD II', 'JOHN OF GAUNT',
       'HENRY BOLINGBROKE', 'THOMAS MOWBRAY', 'DUCHESS', 'Lord Marshal',
       'DUKE OF AUMERLE', 'First Herald', 'Second Herald', 'GREEN',
       'BUSHY', 'DUKE OF YORK', 'QUEEN', 'NORTHUMBERLAND', 'LORD ROSS',
       'LORD WILLOUGHBY', 'Servant', 'BAGOT', 'HENRY PERCY',
       'LORD BERKELEY', 'Captain', 'EARL OF SALISBURY',
       'BISHOP OF CARLISLE', 'SIR STEPHEN SCROOP', 'Lady', 'Gardener',
       'GARDENER', 'LORD FITZWATER', 'Lord', 'DUKE OF SURREY', 'Abbot',
       'EXTON', 'Groom', 'Keeper', 'SAMPSON', 'GREGORY', 'ABRAHAM',
       'BENVOLIO', 'TYBALT', 'CAPULET', 'LADY CAPULET', 'MONTAGUE',
       'LADY MONTAGUE', 'PRINCE', 'ROMEO', 'PARIS', 'Nurse', 'JULIET',
       'MERCUTIO', 'First Servant', 'Second Servant', 'Second Capulet',
       'Chorus', 'FRIAR LAURENCE', 'PETER', 'NURSE', 'LADY  CAPULET',
       'First Musician', 'Second Musician', 'Musician', 'Third Musician',
       'BALTHASAR', 'Apothecary', 'FRIAR JOHN', 'PAGE', 'First Watchman',
       'Second Watchman', 'Third Watchman', 'WARWICK', 'EDWARD',
       'RICHARD', 'KING HENRY VI', 'CLIFFORD', 'WESTMORELAND', 'EXETER',
       'JOHN MORTIMER', 'RUTLAND', 'Tutor', 'GEORGE', 'Son', 'Father',
       'First Keeper', 'Second Keeper', 'LADY GREY', 'Nobleman',
       'KING LEWIS XI', 'BONA', 'Post', 'SOMERSET', 'Huntsman', 'Mayor',
       'Soldier', 'First Messenger', 'ARCHIDAMUS', 'CAMILLO', 'POLIXENES',
       'LEONTES', 'HERMIONE', 'MAMILLIUS', 'First Lady', 'Second Lady',
       'ANTIGONUS', 'Gaoler', 'PAULINA', 'EMILIA', 'CLEOMENES', 'DION',
       'Mariner', 'Shepherd', 'Clown', 'Time', 'AUTOLYCUS', 'FLORIZEL',
       'PERDITA', 'DORCAS', 'MOPSA', 'Shepard', 'First Gentleman',
       'Second Gentleman', 'Third Gentleman', 'DUKE VINCENTIO', 'ESCALUS',
       'ANGELO', 'DUKE', 'LUCIO', 'MISTRESS OVERDONE', 'POMPEY',
       'CLAUDIO', 'Provost', 'FRIAR THOMAS', 'ISABELLA', 'FRANCISCA',
       'ELBOW', 'FROTH', 'POMPHEY', 'Justice', 'MARIANA', 'ABHORSON',
       'BARNARDINE', 'FRIAR PETER', 'SLY', 'Hostess', 'First Huntsman',
       'Second Huntsman', 'Players', 'A Player', 'Third Servant', 'ALL',
       'LUCENTIO', 'TRANIO', 'BAPTISTA', 'GREMIO', 'HORTENSIO',
       'KATHARINA', 'HORTENSIA', 'BIANCA', 'BIONDELLO', 'PETRUCHIO',
       'GRUMIO', 'KATARINA', 'CURTIS', 'NATHANIEL', 'PHILIP', 'JOSEPH',
       'NICHOLAS', 'ALL SERVING-MEN', 'Pedant', 'Tailor', 'VINCENTIO',
       'Widow', 'Master', 'Boatswain', 'ALONSO', 'ANTONIO', 'GONZALO',
       'SEBASTIAN', 'Mariners', 'MIRANDA', 'PROSPERO', 'ARIEL', 'CALIBAN',
       'FERDINAND', 'ADRIAN', 'FRANCISCO'], dtype=object)
\end{Verbatim}
\end{tcolorbox}
        
    Some of the names are in all caps, while some are in sentence case. I'll
adjust so they're all in sentence case.

    \begin{tcolorbox}[breakable, size=fbox, boxrule=1pt, pad at break*=1mm,colback=cellbackground, colframe=cellborder]
\prompt{In}{incolor}{112}{\boxspacing}
\begin{Verbatim}[commandchars=\\\{\}]
\PY{n}{text}\PY{p}{[}\PY{l+s+s1}{\PYZsq{}}\PY{l+s+s1}{Speaker}\PY{l+s+s1}{\PYZsq{}}\PY{p}{]} \PY{o}{=} \PY{n}{text}\PY{p}{[}\PY{l+s+s1}{\PYZsq{}}\PY{l+s+s1}{Speaker}\PY{l+s+s1}{\PYZsq{}}\PY{p}{]}\PY{o}{.}\PY{n}{str}\PY{o}{.}\PY{n}{title}\PY{p}{(}\PY{p}{)}
\end{Verbatim}
\end{tcolorbox}

    Through Googling, I found a table of all characters in all Shakepeare
plays, along with the number of lines spoken by that character and the
play they appear in. When I tried to scrape the table using the
\texttt{read\_html} function from \texttt{pandas}, I received a 403
error code, indicating forbidden access. Fortunately, I was able to take
a much cruder approach and copy/paste the contents of the table into an
Excel file, which I've committed to my branch/folder for this homework
assignment. This dataset comes from playshakespeare.com, specifically
this page:
\url{https://www.playshakespeare.com/study/complete-shakespeare-character-list}.
All credit goes to the original compilers of the data.

    \begin{tcolorbox}[breakable, size=fbox, boxrule=1pt, pad at break*=1mm,colback=cellbackground, colframe=cellborder]
\prompt{In}{incolor}{12}{\boxspacing}
\begin{Verbatim}[commandchars=\\\{\}]
\PY{c+c1}{\PYZsh{}Unsuccessful scraping attempt:}
\PY{c+c1}{\PYZsh{}characters = pd.read\PYZus{}html(\PYZsq{}https://www.playshakespeare.com/study/complete\PYZhy{}shakespeare\PYZhy{}character\PYZhy{}list\PYZsq{})}
\end{Verbatim}
\end{tcolorbox}

    \begin{tcolorbox}[breakable, size=fbox, boxrule=1pt, pad at break*=1mm,colback=cellbackground, colframe=cellborder]
\prompt{In}{incolor}{13}{\boxspacing}
\begin{Verbatim}[commandchars=\\\{\}]
\PY{n}{characters} \PY{o}{=} \PY{n}{pd}\PY{o}{.}\PY{n}{read\PYZus{}csv}\PY{p}{(}\PY{l+s+s1}{\PYZsq{}}\PY{l+s+s1}{shakespeare\PYZus{}characters.csv}\PY{l+s+s1}{\PYZsq{}}\PY{p}{)}
\end{Verbatim}
\end{tcolorbox}

    \begin{tcolorbox}[breakable, size=fbox, boxrule=1pt, pad at break*=1mm,colback=cellbackground, colframe=cellborder]
\prompt{In}{incolor}{14}{\boxspacing}
\begin{Verbatim}[commandchars=\\\{\}]
\PY{n}{characters}\PY{o}{.}\PY{n}{head}\PY{p}{(}\PY{l+m+mi}{10}\PY{p}{)}
\end{Verbatim}
\end{tcolorbox}

            \begin{tcolorbox}[breakable, size=fbox, boxrule=.5pt, pad at break*=1mm, opacityfill=0]
\prompt{Out}{outcolor}{14}{\boxspacing}
\begin{Verbatim}[commandchars=\\\{\}]
                                   cter Lines                       Play
0                King of France (KING.)   373  All's Well That Ends Well
1              Duke of Florence (DUKE.)    19  All's Well That Ends Well
2  Bertram, Count of Roussillion (BER.)   240  All's Well That Ends Well
3                          Lafew (LAF.)   146  All's Well That Ends Well
4                       Parolles (PAR.)   184  All's Well That Ends Well
5                       Rinaldo (STEW.)    23  All's Well That Ends Well
6                        Lavatch (CLO.)    74  All's Well That Ends Well
7               Countess's Page (PAGE.)     1  All's Well That Ends Well
8                     Gentleman (GENT.)    22  All's Well That Ends Well
9      Countess of Roussillion (COUNT.)   220  All's Well That Ends Well
\end{Verbatim}
\end{tcolorbox}
        
    I need to remove the notations in parentheses in order to properly join
my tables, so I'll use a string method.

    \begin{tcolorbox}[breakable, size=fbox, boxrule=1pt, pad at break*=1mm,colback=cellbackground, colframe=cellborder]
\prompt{In}{incolor}{15}{\boxspacing}
\begin{Verbatim}[commandchars=\\\{\}]
\PY{n}{characters}\PY{p}{[}\PY{p}{[}\PY{l+s+s1}{\PYZsq{}}\PY{l+s+s1}{cter\PYZus{}clean}\PY{l+s+s1}{\PYZsq{}}\PY{p}{,}\PY{l+s+s1}{\PYZsq{}}\PY{l+s+s1}{unneeded}\PY{l+s+s1}{\PYZsq{}}\PY{p}{]}\PY{p}{]} \PY{o}{=} \PY{n}{characters}\PY{p}{[}\PY{l+s+s1}{\PYZsq{}}\PY{l+s+s1}{cter}\PY{l+s+s1}{\PYZsq{}}\PY{p}{]}\PY{o}{.}\PY{n}{str}\PY{o}{.}\PY{n}{split}\PY{p}{(}\PY{l+s+s1}{\PYZsq{}}\PY{l+s+s1}{ }\PY{l+s+s1}{\PYZbs{}}\PY{l+s+s1}{(}\PY{l+s+s1}{\PYZsq{}}\PY{p}{,} \PY{l+m+mi}{1}\PY{p}{,} \PY{n}{expand} \PY{o}{=} \PY{k+kc}{True}\PY{p}{)}
\end{Verbatim}
\end{tcolorbox}

    \begin{tcolorbox}[breakable, size=fbox, boxrule=1pt, pad at break*=1mm,colback=cellbackground, colframe=cellborder]
\prompt{In}{incolor}{16}{\boxspacing}
\begin{Verbatim}[commandchars=\\\{\}]
\PY{n}{characters}\PY{o}{.}\PY{n}{head}\PY{p}{(}\PY{l+m+mi}{5}\PY{p}{)}
\end{Verbatim}
\end{tcolorbox}

            \begin{tcolorbox}[breakable, size=fbox, boxrule=.5pt, pad at break*=1mm, opacityfill=0]
\prompt{Out}{outcolor}{16}{\boxspacing}
\begin{Verbatim}[commandchars=\\\{\}]
                                   cter Lines                       Play  \textbackslash{}
0                King of France (KING.)   373  All's Well That Ends Well
1              Duke of Florence (DUKE.)    19  All's Well That Ends Well
2  Bertram, Count of Roussillion (BER.)   240  All's Well That Ends Well
3                          Lafew (LAF.)   146  All's Well That Ends Well
4                       Parolles (PAR.)   184  All's Well That Ends Well

                      cter\_clean unneeded
0                 King of France   KING.)
1               Duke of Florence   DUKE.)
2  Bertram, Count of Roussillion    BER.)
3                          Lafew    LAF.)
4                       Parolles    PAR.)
\end{Verbatim}
\end{tcolorbox}
        
    \begin{tcolorbox}[breakable, size=fbox, boxrule=1pt, pad at break*=1mm,colback=cellbackground, colframe=cellborder]
\prompt{In}{incolor}{17}{\boxspacing}
\begin{Verbatim}[commandchars=\\\{\}]
\PY{c+c1}{\PYZsh{}Drop unneeded columns}
\PY{n}{characters} \PY{o}{=} \PY{n}{characters}\PY{p}{[}\PY{p}{[}\PY{l+s+s1}{\PYZsq{}}\PY{l+s+s1}{cter\PYZus{}clean}\PY{l+s+s1}{\PYZsq{}}\PY{p}{,} \PY{l+s+s1}{\PYZsq{}}\PY{l+s+s1}{Play}\PY{l+s+s1}{\PYZsq{}}\PY{p}{,} \PY{l+s+s1}{\PYZsq{}}\PY{l+s+s1}{Lines}\PY{l+s+s1}{\PYZsq{}}\PY{p}{]}\PY{p}{]}
\end{Verbatim}
\end{tcolorbox}

    \begin{tcolorbox}[breakable, size=fbox, boxrule=1pt, pad at break*=1mm,colback=cellbackground, colframe=cellborder]
\prompt{In}{incolor}{18}{\boxspacing}
\begin{Verbatim}[commandchars=\\\{\}]
\PY{n}{text\PYZus{}play} \PY{o}{=} \PY{n}{pd}\PY{o}{.}\PY{n}{merge}\PY{p}{(}\PY{n}{text}\PY{p}{,} \PY{n}{characters}\PY{p}{,} \PY{n}{left\PYZus{}on} \PY{o}{=} \PY{p}{[}\PY{l+s+s1}{\PYZsq{}}\PY{l+s+s1}{Speaker}\PY{l+s+s1}{\PYZsq{}}\PY{p}{]}\PY{p}{,} \PY{n}{right\PYZus{}on} \PY{o}{=} \PY{p}{[}\PY{l+s+s1}{\PYZsq{}}\PY{l+s+s1}{cter\PYZus{}clean}\PY{l+s+s1}{\PYZsq{}}\PY{p}{]}\PY{p}{,} \PY{n}{how} \PY{o}{=} \PY{l+s+s1}{\PYZsq{}}\PY{l+s+s1}{left}\PY{l+s+s1}{\PYZsq{}}\PY{p}{)}
\end{Verbatim}
\end{tcolorbox}

    This probably achieved a decent amount of matching, but checking the
head and tail of the data set reveals two problems:

\begin{enumerate}
\def\labelenumi{\arabic{enumi})}
\item
  Some lines are spoken by characters with ambiguous names (``All'' for
  the ensemble in response to the First Citizen in Richard III), and so
  have no match in the character data set.
\item
  Shakespeare reused some character names across multiple plays. For
  example, there are characters named Antonio in The Merchant of Venice,
  Much Ado About Nothing, The Tempest, Twelfth Night, and The Two
  Gentlemen of Verona, and simply joining on character name is not
  enough to specify which one of those plays corresponds to that
  particular Antonio.
\end{enumerate}

Issue (1) has a fairly straightforward-ish solution which should
appropriately label many of the NAs. If I assume that these lines are
not in a random order, and consist of the full (or partial) dialogue of
each play in the order that dialogue is spoken (i.e.~a line from Romeo
and Juliet is surrounded by other lines from Romeo and Juliet, and is
not intermingled with lines from The Taming of the Shrew or Hamlet or
King Lear), then I can compare the values of each NA line to the closest
non-NA values preceding and following it, and use those to assign a
play. For example, line 2, spoken by the ensemble (labeled ``All'') is
preceded by a line identified as matching a character in Richard III,
and is followed by a line matching a character in Richard III, and so is
almost certainly itself a line from Richard III. I can also fill in the
character name with the value from the `Speaker' column in cases where
that would be necessary.

(Note: The example above assumes that the initial match of Richard III
for those lines was accurate. I don't think it was accurate, but I'm
proceeding as if it was. I'll address this further toward the end of
this part.)

    \begin{tcolorbox}[breakable, size=fbox, boxrule=1pt, pad at break*=1mm,colback=cellbackground, colframe=cellborder]
\prompt{In}{incolor}{19}{\boxspacing}
\begin{Verbatim}[commandchars=\\\{\}]
\PY{c+c1}{\PYZsh{}One column has, for each row, the next non\PYZhy{}NA value, while the other has the previous non\PYZhy{}NA value.}
\PY{n}{text\PYZus{}play}\PY{p}{[}\PY{l+s+s1}{\PYZsq{}}\PY{l+s+s1}{prev\PYZus{}play}\PY{l+s+s1}{\PYZsq{}}\PY{p}{]} \PY{o}{=} \PY{n}{text\PYZus{}play}\PY{p}{[}\PY{l+s+s1}{\PYZsq{}}\PY{l+s+s1}{Play}\PY{l+s+s1}{\PYZsq{}}\PY{p}{]}\PY{o}{.}\PY{n}{bfill}\PY{p}{(}\PY{p}{)}
\PY{n}{text\PYZus{}play}\PY{p}{[}\PY{l+s+s1}{\PYZsq{}}\PY{l+s+s1}{next\PYZus{}play}\PY{l+s+s1}{\PYZsq{}}\PY{p}{]} \PY{o}{=} \PY{n}{text\PYZus{}play}\PY{p}{[}\PY{l+s+s1}{\PYZsq{}}\PY{l+s+s1}{Play}\PY{l+s+s1}{\PYZsq{}}\PY{p}{]}\PY{o}{.}\PY{n}{ffill}\PY{p}{(}\PY{p}{)}
\end{Verbatim}
\end{tcolorbox}

    \begin{tcolorbox}[breakable, size=fbox, boxrule=1pt, pad at break*=1mm,colback=cellbackground, colframe=cellborder]
\prompt{In}{incolor}{20}{\boxspacing}
\begin{Verbatim}[commandchars=\\\{\}]
\PY{k+kn}{import} \PY{n+nn}{numpy} \PY{k}{as} \PY{n+nn}{np}
\PY{n}{text\PYZus{}play}\PY{p}{[}\PY{l+s+s1}{\PYZsq{}}\PY{l+s+s1}{Play}\PY{l+s+s1}{\PYZsq{}}\PY{p}{]} \PY{o}{=} \PY{n}{np}\PY{o}{.}\PY{n}{where}\PY{p}{(}\PY{n}{text\PYZus{}play}\PY{p}{[}\PY{l+s+s1}{\PYZsq{}}\PY{l+s+s1}{prev\PYZus{}play}\PY{l+s+s1}{\PYZsq{}}\PY{p}{]} \PY{o}{==} \PY{n}{text\PYZus{}play}\PY{p}{[}\PY{l+s+s1}{\PYZsq{}}\PY{l+s+s1}{next\PYZus{}play}\PY{l+s+s1}{\PYZsq{}}\PY{p}{]}\PY{p}{,}
                             \PY{n}{text\PYZus{}play}\PY{p}{[}\PY{l+s+s1}{\PYZsq{}}\PY{l+s+s1}{prev\PYZus{}play}\PY{l+s+s1}{\PYZsq{}}\PY{p}{]}\PY{p}{,}
                             \PY{n}{text\PYZus{}play}\PY{p}{[}\PY{l+s+s1}{\PYZsq{}}\PY{l+s+s1}{Play}\PY{l+s+s1}{\PYZsq{}}\PY{p}{]}\PY{p}{)}
\end{Verbatim}
\end{tcolorbox}

    To address issue 2 I described above (the ``multiple plays
-\textgreater{} 1 character name'' problem), I tried the following:
since the result of the left join ends up duplicating any rows in the
original text data frame which match more than one play, that means that
some of the line numbers I generated appear more than once. I can find,
for each row, the number of times that line number appears in the data
set, set all the values where count(line number) != 1 as NA, and then do
the same comparisons as above.

For example, the selection of rows below shows a conversation between
three characters named Gonzalo, Sebastian, and Antonio:

    \begin{tcolorbox}[breakable, size=fbox, boxrule=1pt, pad at break*=1mm,colback=cellbackground, colframe=cellborder]
\prompt{In}{incolor}{21}{\boxspacing}
\begin{Verbatim}[commandchars=\\\{\}]
\PY{n}{text\PYZus{}play}\PY{p}{[}\PY{l+m+mi}{9972}\PY{p}{:}\PY{l+m+mi}{9983}\PY{p}{]}
\end{Verbatim}
\end{tcolorbox}

            \begin{tcolorbox}[breakable, size=fbox, boxrule=.5pt, pad at break*=1mm, opacityfill=0]
\prompt{Out}{outcolor}{21}{\boxspacing}
\begin{Verbatim}[commandchars=\\\{\}]
      Line\_Number    Speaker  \textbackslash{}
9972         7205  Sebastian
9973         7205  Sebastian
9974         7206    Gonzalo
9975         7207  Sebastian
9976         7207  Sebastian
9977         7208    Antonio
9978         7208    Antonio
9979         7208    Antonio
9980         7208    Antonio
9981         7208    Antonio
9982         7209    Gonzalo

                                                         Text  \textbackslash{}
9972
An it had not fallen flat-long.
9973
An it had not fallen flat-long.
9974  You are gentlemen of brave metal; you would lift\textbackslash{}nthe moon out of her
sphere, if she would continue\textbackslash{}nin it five weeks without changing.
9975
We would so, and then go a bat-fowling.
9976
We would so, and then go a bat-fowling.
9977
Nay, good my lord, be not angry.
9978
Nay, good my lord, be not angry.
9979
Nay, good my lord, be not angry.
9980
Nay, good my lord, be not angry.
9981
Nay, good my lord, be not angry.
9982                        No, I warrant you; I will not adventure\textbackslash{}nmy
discretion so weakly. Will you laugh\textbackslash{}nme asleep, for I am very heavy?

     cter\_clean                         Play Lines  \textbackslash{}
9972  Sebastian                  The Tempest   114
9973  Sebastian                Twelfth Night    98
9974    Gonzalo                  The Tempest   135
9975  Sebastian                  The Tempest   114
9976  Sebastian                Twelfth Night    98
9977    Antonio       The Merchant of Venice   189
9978    Antonio       Much Ado About Nothing    44
9979    Antonio                  The Tempest   142
9980    Antonio                Twelfth Night   105
9981    Antonio  The Two Gentlemen of Verona    35
9982    Gonzalo                  The Tempest   135

                        prev\_play                    next\_play
9972                  The Tempest                  The Tempest
9973                Twelfth Night                Twelfth Night
9974                  The Tempest                  The Tempest
9975                  The Tempest                  The Tempest
9976                Twelfth Night                Twelfth Night
9977       The Merchant of Venice       The Merchant of Venice
9978       Much Ado About Nothing       Much Ado About Nothing
9979                  The Tempest                  The Tempest
9980                Twelfth Night                Twelfth Night
9981  The Two Gentlemen of Verona  The Two Gentlemen of Verona
9982                  The Tempest                  The Tempest
\end{Verbatim}
\end{tcolorbox}
        
    Sebastian and Antonio each have at least two matches, but Gonzalo only
matches one play: The Tempest. As it turns out, Sebastian and Antonio
also have The Tempest as a possible match for their dialogue, and since
they are talking to a character who only appears in that play, we can
assume that these lines correspond to The Tempest, and not one of the
other plays which matched the character names.

To do this, I did\ldots well, you'll see. As a result of the left join
matching multiple plays to the same line of dialogue, many of the line
numbers (like 7208 above) appear more than once in the data set.

First, I used the \texttt{value\_counts} function to label each row with
the number of times that row's line number appears. I then merged the
line number counts into the text\_play dataframe I've been working with.
Then I replaced all the count(line number) values not equal to 1 with
\texttt{np.nan}, and replaced the values equal to 1 with the
corresponding play (to get the play names for the rows with unique
matches). I then dropped the \texttt{prev\_play} and \texttt{next\_play}
columns and re-instantiated them using the column I just constructed
with the unique line counts.

Basically, for each line of dialogue which matches multiple plays, this
finds the closest unique play before and after it. The same way that I
assumed a line where the play was NA but surrounded by lines matching
characters in Richard III was itself from Richard III, here I assume
that if the closest unique characters to a multiple-matched are from The
Tempest, then the correct match from those multiple matches must also be
The Tempest, even though that speaker's name appears more than once in
the data set.

    \begin{tcolorbox}[breakable, size=fbox, boxrule=1pt, pad at break*=1mm,colback=cellbackground, colframe=cellborder]
\prompt{In}{incolor}{22}{\boxspacing}
\begin{Verbatim}[commandchars=\\\{\}]
\PY{n}{line\PYZus{}counts} \PY{o}{=} \PY{n}{text\PYZus{}play}\PY{p}{[}\PY{l+s+s1}{\PYZsq{}}\PY{l+s+s1}{Line\PYZus{}Number}\PY{l+s+s1}{\PYZsq{}}\PY{p}{]}\PY{o}{.}\PY{n}{value\PYZus{}counts}\PY{p}{(}\PY{p}{)}
\end{Verbatim}
\end{tcolorbox}

    \begin{tcolorbox}[breakable, size=fbox, boxrule=1pt, pad at break*=1mm,colback=cellbackground, colframe=cellborder]
\prompt{In}{incolor}{23}{\boxspacing}
\begin{Verbatim}[commandchars=\\\{\}]
\PY{n}{line\PYZus{}counts} \PY{o}{=} \PY{n}{pd}\PY{o}{.}\PY{n}{DataFrame}\PY{p}{(}\PY{n}{line\PYZus{}counts}\PY{p}{)}
\end{Verbatim}
\end{tcolorbox}

    \begin{tcolorbox}[breakable, size=fbox, boxrule=1pt, pad at break*=1mm,colback=cellbackground, colframe=cellborder]
\prompt{In}{incolor}{24}{\boxspacing}
\begin{Verbatim}[commandchars=\\\{\}]
\PY{n}{line\PYZus{}counts} \PY{o}{=} \PY{n}{line\PYZus{}counts}\PY{o}{.}\PY{n}{reset\PYZus{}index}\PY{p}{(}\PY{p}{)}\PY{o}{.}\PY{n}{rename}\PY{p}{(}\PY{n}{columns} \PY{o}{=} \PY{p}{\PYZob{}}\PY{l+s+s1}{\PYZsq{}}\PY{l+s+s1}{Line\PYZus{}Number}\PY{l+s+s1}{\PYZsq{}}\PY{p}{:} \PY{l+s+s2}{\PYZdq{}}\PY{l+s+s2}{line\PYZus{}counts}\PY{l+s+s2}{\PYZdq{}}\PY{p}{,}
                                                          \PY{l+s+s1}{\PYZsq{}}\PY{l+s+s1}{index}\PY{l+s+s1}{\PYZsq{}}\PY{p}{:} \PY{l+s+s1}{\PYZsq{}}\PY{l+s+s1}{Line\PYZus{}Number}\PY{l+s+s1}{\PYZsq{}}\PY{p}{\PYZcb{}}\PY{p}{)}
\end{Verbatim}
\end{tcolorbox}

    \begin{tcolorbox}[breakable, size=fbox, boxrule=1pt, pad at break*=1mm,colback=cellbackground, colframe=cellborder]
\prompt{In}{incolor}{25}{\boxspacing}
\begin{Verbatim}[commandchars=\\\{\}]
\PY{n}{text\PYZus{}play} \PY{o}{=} \PY{n}{pd}\PY{o}{.}\PY{n}{merge}\PY{p}{(}\PY{n}{text\PYZus{}play}\PY{p}{,}
                     \PY{n}{line\PYZus{}counts}\PY{p}{,}
                     \PY{n}{left\PYZus{}on} \PY{o}{=} \PY{p}{[}\PY{l+s+s1}{\PYZsq{}}\PY{l+s+s1}{Line\PYZus{}Number}\PY{l+s+s1}{\PYZsq{}}\PY{p}{]}\PY{p}{,}
                     \PY{n}{right\PYZus{}on} \PY{o}{=} \PY{p}{[}\PY{l+s+s1}{\PYZsq{}}\PY{l+s+s1}{Line\PYZus{}Number}\PY{l+s+s1}{\PYZsq{}}\PY{p}{]}\PY{p}{,} \PY{n}{how} \PY{o}{=} \PY{l+s+s1}{\PYZsq{}}\PY{l+s+s1}{left}\PY{l+s+s1}{\PYZsq{}}\PY{p}{)}
\end{Verbatim}
\end{tcolorbox}

    \begin{tcolorbox}[breakable, size=fbox, boxrule=1pt, pad at break*=1mm,colback=cellbackground, colframe=cellborder]
\prompt{In}{incolor}{26}{\boxspacing}
\begin{Verbatim}[commandchars=\\\{\}]
\PY{n}{text\PYZus{}play}\PY{p}{[}\PY{l+s+s1}{\PYZsq{}}\PY{l+s+s1}{linecounts\PYZus{}play}\PY{l+s+s1}{\PYZsq{}}\PY{p}{]} \PY{o}{=} \PY{n}{np}\PY{o}{.}\PY{n}{where}\PY{p}{(}\PY{n}{text\PYZus{}play}\PY{p}{[}\PY{l+s+s1}{\PYZsq{}}\PY{l+s+s1}{line\PYZus{}counts}\PY{l+s+s1}{\PYZsq{}}\PY{p}{]} \PY{o}{!=} \PY{l+m+mi}{1}\PY{p}{,}
                                    \PY{n}{np}\PY{o}{.}\PY{n}{nan}\PY{p}{,}
                                    \PY{n}{text\PYZus{}play}\PY{p}{[}\PY{l+s+s1}{\PYZsq{}}\PY{l+s+s1}{Play}\PY{l+s+s1}{\PYZsq{}}\PY{p}{]}\PY{p}{)}
\end{Verbatim}
\end{tcolorbox}

    \begin{tcolorbox}[breakable, size=fbox, boxrule=1pt, pad at break*=1mm,colback=cellbackground, colframe=cellborder]
\prompt{In}{incolor}{27}{\boxspacing}
\begin{Verbatim}[commandchars=\\\{\}]
\PY{n}{text\PYZus{}play} \PY{o}{=} \PY{n}{text\PYZus{}play}\PY{o}{.}\PY{n}{drop}\PY{p}{(}\PY{n}{columns} \PY{o}{=} \PY{p}{[}\PY{l+s+s1}{\PYZsq{}}\PY{l+s+s1}{prev\PYZus{}play}\PY{l+s+s1}{\PYZsq{}}\PY{p}{,} \PY{l+s+s1}{\PYZsq{}}\PY{l+s+s1}{next\PYZus{}play}\PY{l+s+s1}{\PYZsq{}}\PY{p}{]}\PY{p}{)}
\end{Verbatim}
\end{tcolorbox}

    \begin{tcolorbox}[breakable, size=fbox, boxrule=1pt, pad at break*=1mm,colback=cellbackground, colframe=cellborder]
\prompt{In}{incolor}{28}{\boxspacing}
\begin{Verbatim}[commandchars=\\\{\}]
\PY{n}{text\PYZus{}play}\PY{p}{[}\PY{l+s+s1}{\PYZsq{}}\PY{l+s+s1}{prev\PYZus{}play}\PY{l+s+s1}{\PYZsq{}}\PY{p}{]} \PY{o}{=} \PY{n}{text\PYZus{}play}\PY{p}{[}\PY{l+s+s1}{\PYZsq{}}\PY{l+s+s1}{linecounts\PYZus{}play}\PY{l+s+s1}{\PYZsq{}}\PY{p}{]}\PY{o}{.}\PY{n}{bfill}\PY{p}{(}\PY{p}{)}
\PY{n}{text\PYZus{}play}\PY{p}{[}\PY{l+s+s1}{\PYZsq{}}\PY{l+s+s1}{next\PYZus{}play}\PY{l+s+s1}{\PYZsq{}}\PY{p}{]} \PY{o}{=} \PY{n}{text\PYZus{}play}\PY{p}{[}\PY{l+s+s1}{\PYZsq{}}\PY{l+s+s1}{linecounts\PYZus{}play}\PY{l+s+s1}{\PYZsq{}}\PY{p}{]}\PY{o}{.}\PY{n}{ffill}\PY{p}{(}\PY{p}{)}
\end{Verbatim}
\end{tcolorbox}

    Now I create a new column (creatively titled \texttt{Play1}). For a
given row, if the previous unique non-NA value of the \texttt{Play}
column equals the next unique non-NA value, then this column equals that
value. Otherwise, it just equals the \texttt{Play} column. I also fill
in the missing values in the \texttt{Play} column with the values in the
\texttt{prev\_play} column, for merging purposes later.

    \begin{tcolorbox}[breakable, size=fbox, boxrule=1pt, pad at break*=1mm,colback=cellbackground, colframe=cellborder]
\prompt{In}{incolor}{29}{\boxspacing}
\begin{Verbatim}[commandchars=\\\{\}]
\PY{n}{text\PYZus{}play}\PY{p}{[}\PY{l+s+s1}{\PYZsq{}}\PY{l+s+s1}{Play1}\PY{l+s+s1}{\PYZsq{}}\PY{p}{]} \PY{o}{=} \PY{n}{np}\PY{o}{.}\PY{n}{where}\PY{p}{(}\PY{n}{text\PYZus{}play}\PY{p}{[}\PY{l+s+s1}{\PYZsq{}}\PY{l+s+s1}{prev\PYZus{}play}\PY{l+s+s1}{\PYZsq{}}\PY{p}{]} \PY{o}{==} \PY{n}{text\PYZus{}play}\PY{p}{[}\PY{l+s+s1}{\PYZsq{}}\PY{l+s+s1}{next\PYZus{}play}\PY{l+s+s1}{\PYZsq{}}\PY{p}{]}\PY{p}{,}
                                    \PY{n}{text\PYZus{}play}\PY{p}{[}\PY{l+s+s1}{\PYZsq{}}\PY{l+s+s1}{prev\PYZus{}play}\PY{l+s+s1}{\PYZsq{}}\PY{p}{]}\PY{p}{,}
                                    \PY{n}{text\PYZus{}play}\PY{p}{[}\PY{l+s+s1}{\PYZsq{}}\PY{l+s+s1}{Play}\PY{l+s+s1}{\PYZsq{}}\PY{p}{]}\PY{p}{)}
\end{Verbatim}
\end{tcolorbox}

    \begin{tcolorbox}[breakable, size=fbox, boxrule=1pt, pad at break*=1mm,colback=cellbackground, colframe=cellborder]
\prompt{In}{incolor}{30}{\boxspacing}
\begin{Verbatim}[commandchars=\\\{\}]
\PY{n}{text\PYZus{}play}\PY{p}{[}\PY{l+s+s1}{\PYZsq{}}\PY{l+s+s1}{Play}\PY{l+s+s1}{\PYZsq{}}\PY{p}{]} \PY{o}{=} \PY{n}{np}\PY{o}{.}\PY{n}{where}\PY{p}{(}\PY{n}{text\PYZus{}play}\PY{p}{[}\PY{l+s+s1}{\PYZsq{}}\PY{l+s+s1}{Play}\PY{l+s+s1}{\PYZsq{}}\PY{p}{]}\PY{o}{.}\PY{n}{isna}\PY{p}{(}\PY{p}{)}\PY{p}{,}
                             \PY{n}{text\PYZus{}play}\PY{p}{[}\PY{l+s+s1}{\PYZsq{}}\PY{l+s+s1}{prev\PYZus{}play}\PY{l+s+s1}{\PYZsq{}}\PY{p}{]}\PY{p}{,}
                             \PY{n}{text\PYZus{}play}\PY{p}{[}\PY{l+s+s1}{\PYZsq{}}\PY{l+s+s1}{Play}\PY{l+s+s1}{\PYZsq{}}\PY{p}{]}\PY{p}{)}
\end{Verbatim}
\end{tcolorbox}

    Now, for each value of \texttt{Line\_Number}, I subset those rows, then
create a series containing those rows' values of Play, prev\_play,
linecounts\_play, next\_play, and Play1 (basically all of the additional
columns and indices I created to try and fix the metadata). Then I find
how often each value in that series occurs using value\_count, and then
drop all the items in the series which occur at the same frequency as
the modal value (since I assume that the modal play is correct, a VERY
big assumption). Then I append the rows containing the combinations of
line number and play title to be dropped to a list (creatively titled
\texttt{drop\_list1}).

Once the for loop below is finished, I then concatenate the list of line
number/play combos I want to drop, convert it to a data frame, and then
performan an anti-join between the text\_play dataframe and that
dataframe.

    \begin{tcolorbox}[breakable, size=fbox, boxrule=1pt, pad at break*=1mm,colback=cellbackground, colframe=cellborder]
\prompt{In}{incolor}{32}{\boxspacing}
\begin{Verbatim}[commandchars=\\\{\}]
\PY{n}{drop\PYZus{}list1} \PY{o}{=} \PY{p}{[}\PY{p}{]}
\PY{k}{for} \PY{n}{ii} \PY{o+ow}{in} \PY{n}{text\PYZus{}play}\PY{p}{[}\PY{l+s+s1}{\PYZsq{}}\PY{l+s+s1}{Line\PYZus{}Number}\PY{l+s+s1}{\PYZsq{}}\PY{p}{]}\PY{o}{.}\PY{n}{unique}\PY{p}{(}\PY{p}{)}\PY{p}{:}
    \PY{c+c1}{\PYZsh{}if text\PYZus{}play1[text\PYZus{}play1[\PYZsq{}Line\PYZus{}Number\PYZsq{}]==ii][\PYZsq{}line\PYZus{}counts\PYZsq{}].unique() != 1:}
    
    \PY{c+c1}{\PYZsh{}Select rows where line number = ii}
    \PY{n}{rows} \PY{o}{=} \PY{n}{text\PYZus{}play}\PY{p}{[}\PY{n}{text\PYZus{}play}\PY{p}{[}\PY{l+s+s1}{\PYZsq{}}\PY{l+s+s1}{Line\PYZus{}Number}\PY{l+s+s1}{\PYZsq{}}\PY{p}{]}\PY{o}{==}\PY{n}{ii}\PY{p}{]}
    
    \PY{c+c1}{\PYZsh{}Select these 5 columns, and stack them into a single series}
    \PY{n}{options} \PY{o}{=} \PY{n}{rows}\PY{p}{[}\PY{p}{[}\PY{l+s+s1}{\PYZsq{}}\PY{l+s+s1}{Play}\PY{l+s+s1}{\PYZsq{}}\PY{p}{,}
                    \PY{l+s+s1}{\PYZsq{}}\PY{l+s+s1}{linecounts\PYZus{}play}\PY{l+s+s1}{\PYZsq{}}\PY{p}{,}
                    \PY{l+s+s1}{\PYZsq{}}\PY{l+s+s1}{prev\PYZus{}play}\PY{l+s+s1}{\PYZsq{}}\PY{p}{,}
                    \PY{l+s+s1}{\PYZsq{}}\PY{l+s+s1}{next\PYZus{}play}\PY{l+s+s1}{\PYZsq{}}\PY{p}{,}
                    \PY{l+s+s1}{\PYZsq{}}\PY{l+s+s1}{Play1}\PY{l+s+s1}{\PYZsq{}}\PY{p}{]}\PY{p}{]}\PY{o}{.}\PY{n}{stack}\PY{p}{(}\PY{p}{)}
    
    \PY{c+c1}{\PYZsh{}Find the value counts and set them up as a dataframe}
    \PY{n}{selection} \PY{o}{=} \PY{n}{pd}\PY{o}{.}\PY{n}{DataFrame}\PY{p}{(}\PY{n}{options}\PY{o}{.}\PY{n}{value\PYZus{}counts}\PY{p}{(}\PY{p}{)}\PY{o}{.}\PY{n}{reset\PYZus{}index}\PY{p}{(}\PY{p}{)}\PY{p}{)}\PY{o}{.}\PY{n}{rename}\PY{p}{(}\PY{n}{columns} \PY{o}{=} \PY{p}{\PYZob{}}\PY{l+s+s1}{\PYZsq{}}\PY{l+s+s1}{index}\PY{l+s+s1}{\PYZsq{}}\PY{p}{:} \PY{l+s+s1}{\PYZsq{}}\PY{l+s+s1}{Play}\PY{l+s+s1}{\PYZsq{}}\PY{p}{,}
                                                                                         \PY{l+m+mi}{0}\PY{p}{:} \PY{l+s+s1}{\PYZsq{}}\PY{l+s+s1}{count}\PY{l+s+s1}{\PYZsq{}}\PY{p}{\PYZcb{}}\PY{p}{)}
    \PY{c+c1}{\PYZsh{}If there is more than 1 pair of line number and play, drop the rows with the max frequency}
    \PY{k}{if} \PY{n+nb}{len}\PY{p}{(}\PY{n}{selection}\PY{p}{)} \PY{o}{\PYZgt{}} \PY{l+m+mi}{1}\PY{p}{:}
        \PY{n}{to\PYZus{}drop} \PY{o}{=} \PY{n}{selection}\PY{p}{[}\PY{n}{selection}\PY{p}{[}\PY{l+s+s1}{\PYZsq{}}\PY{l+s+s1}{count}\PY{l+s+s1}{\PYZsq{}}\PY{p}{]}\PY{o}{!=}\PY{n}{selection}\PY{p}{[}\PY{l+s+s1}{\PYZsq{}}\PY{l+s+s1}{count}\PY{l+s+s1}{\PYZsq{}}\PY{p}{]}\PY{o}{.}\PY{n}{max}\PY{p}{(}\PY{p}{)}\PY{p}{]}
        \PY{c+c1}{\PYZsh{}Add a column of just the line number for the anti join}
        \PY{n}{to\PYZus{}drop}\PY{p}{[}\PY{l+s+s1}{\PYZsq{}}\PY{l+s+s1}{Line\PYZus{}Number}\PY{l+s+s1}{\PYZsq{}}\PY{p}{]} \PY{o}{=} \PY{n}{ii}
        \PY{c+c1}{\PYZsh{}Append to the drop list}
        \PY{n}{drop\PYZus{}list1}\PY{o}{.}\PY{n}{append}\PY{p}{(}\PY{n}{to\PYZus{}drop}\PY{p}{)}
        \PY{c+c1}{\PYZsh{}to\PYZus{}drop = to\PYZus{}drop[[\PYZsq{}Line\PYZus{}Number\PYZsq{}, \PYZsq{}Play\PYZsq{}]].apply(tuple, axis=1)}
        \PY{c+c1}{\PYZsh{}text\PYZus{}play1 = text\PYZus{}play1[\PYZti{}text\PYZus{}play1[[\PYZsq{}Line\PYZus{}Number\PYZsq{},\PYZsq{}Play\PYZsq{}]].isin(to\PYZus{}drop)]}
\end{Verbatim}
\end{tcolorbox}

    \begin{Verbatim}[commandchars=\\\{\}]
<ipython-input-32-0a192a9eb097>:21: SettingWithCopyWarning:
A value is trying to be set on a copy of a slice from a DataFrame.
Try using .loc[row\_indexer,col\_indexer] = value instead

See the caveats in the documentation: https://pandas.pydata.org/pandas-
docs/stable/user\_guide/indexing.html\#returning-a-view-versus-a-copy
  to\_drop['Line\_Number'] = ii
    \end{Verbatim}

    \begin{tcolorbox}[breakable, size=fbox, boxrule=1pt, pad at break*=1mm,colback=cellbackground, colframe=cellborder]
\prompt{In}{incolor}{33}{\boxspacing}
\begin{Verbatim}[commandchars=\\\{\}]
\PY{n}{drop\PYZus{}list1} \PY{o}{=} \PY{n}{pd}\PY{o}{.}\PY{n}{concat}\PY{p}{(}\PY{n}{drop\PYZus{}list1}\PY{p}{)}
\end{Verbatim}
\end{tcolorbox}

    \begin{tcolorbox}[breakable, size=fbox, boxrule=1pt, pad at break*=1mm,colback=cellbackground, colframe=cellborder]
\prompt{In}{incolor}{36}{\boxspacing}
\begin{Verbatim}[commandchars=\\\{\}]
\PY{n}{text\PYZus{}play1} \PY{o}{=} \PY{n}{text\PYZus{}play}\PY{o}{.}\PY{n}{merge}\PY{p}{(}\PY{n}{drop\PYZus{}list1}\PY{p}{,} \PY{n}{on} \PY{o}{=} \PY{p}{[}\PY{l+s+s1}{\PYZsq{}}\PY{l+s+s1}{Line\PYZus{}Number}\PY{l+s+s1}{\PYZsq{}}\PY{p}{,} \PY{l+s+s1}{\PYZsq{}}\PY{l+s+s1}{Play}\PY{l+s+s1}{\PYZsq{}}\PY{p}{]}\PY{p}{,} \PY{n}{how} \PY{o}{=} \PY{l+s+s1}{\PYZsq{}}\PY{l+s+s1}{left}\PY{l+s+s1}{\PYZsq{}}\PY{p}{,} \PY{n}{indicator} \PY{o}{=} \PY{k+kc}{True}\PY{p}{)}
\end{Verbatim}
\end{tcolorbox}

    \begin{tcolorbox}[breakable, size=fbox, boxrule=1pt, pad at break*=1mm,colback=cellbackground, colframe=cellborder]
\prompt{In}{incolor}{37}{\boxspacing}
\begin{Verbatim}[commandchars=\\\{\}]
\PY{n}{text\PYZus{}play1} \PY{o}{=} \PY{n}{text\PYZus{}play1}\PY{o}{.}\PY{n}{loc}\PY{p}{[}\PY{n}{text\PYZus{}play1}\PY{p}{[}\PY{l+s+s1}{\PYZsq{}}\PY{l+s+s1}{\PYZus{}merge}\PY{l+s+s1}{\PYZsq{}}\PY{p}{]} \PY{o}{==} \PY{l+s+s1}{\PYZsq{}}\PY{l+s+s1}{left\PYZus{}only}\PY{l+s+s1}{\PYZsq{}}\PY{p}{]}
\end{Verbatim}
\end{tcolorbox}

    This is the part I don't fully understand. The dataframe I produce at
the end of part 1 has 7,222 rows, and theoretically, the data frame I
produce at the end of part 2 should have that many rows as well.
However, the dataframe \texttt{text\_play1}, which is my final product
for part 2, only has 6987 rows. Granted, if each row (meaning each chunk
of text) is a ``document'', then that represents a loss of approximately
3.2\%. Not great, but I don't think that's a project-ending level of
loss.

Side note: if you can figure out where in this process I lose those 235
rows, please let me know!

The last note I have on this particular part is that while this process
achieves a reasonable degree of matching, I know that not all of the
matches are accurate. This is due to a mismatch between the speaker
identifiers in the original data and the identifiers used by
\url{playshakespeare.com}. This data set might list a character as
``First Citizen'' and have it correspond to ``First Roman Citizen'' in
the play Coriolanus, while the character \url{playshakespeare.com}
identifies as ``First Citizen'' is actually from the play Richard III.
Looking back on what I did for this part (which was largely improvised),
I think I tried to apply some of the same assumptions and ideas we
discussed in class (mainly a super-generalized Markov assumption that
things near each other are likely to be more similar than things farther
apart), but didn't end up doing a lot of \emph{text}-based matching.

Given more time, I probably could have achieved a more accurate set of
matches between text and play using the data available. I tried to use a
list comprehension method to match character names from the character
dataset to lines of dialogue, but was unable to get that to work
(something to do with a list not being hashable).

    \begin{center}\rule{0.5\linewidth}{0.5pt}\end{center}

\hypertarget{part-3}{%
\subsection{Part 3}\label{part-3}}

Pick one or more of the techniques described in this chapter:

\begin{verbatim}
keyword frequency
entity relationships
markov language model
bag-of-words, TF-IDF
semantic embedding
\end{verbatim}

and make a case for a technique to measure how important or interesting
a speaker is. The measure does not have to be both important and
interesting, and you are welcome to come up with another term that
represents ``useful content'', or tells a story (happiest speaker, worst
speaker, etc.)

Whatever you choose, you must

\begin{verbatim}
document how your technique was applied
describe why you believe the technique is a valid approximation or exploration of how important, interesting, etc., a speaker is.
list some possible weaknesses of your method, or ways you expect your assumptions could be violated within the text.
\end{verbatim}

This is mostly about learning to transparently document your decisions,
and iterate on a method for operationalizing useful analyses on text.
Your explanations should be understandable; homeworks will be
peer-reviewed by your fellow students.

    \begin{center}\rule{0.5\linewidth}{0.5pt}\end{center}

First, I select only a few columns from the data set (I don't need a lot
of the columns of plays I used and generated in part 2 for this step).

    \begin{tcolorbox}[breakable, size=fbox, boxrule=1pt, pad at break*=1mm,colback=cellbackground, colframe=cellborder]
\prompt{In}{incolor}{51}{\boxspacing}
\begin{Verbatim}[commandchars=\\\{\}]
\PY{n}{text\PYZus{}play2} \PY{o}{=} \PY{n}{text\PYZus{}play1}\PY{p}{[}\PY{p}{[}\PY{l+s+s1}{\PYZsq{}}\PY{l+s+s1}{Line\PYZus{}Number}\PY{l+s+s1}{\PYZsq{}}\PY{p}{,}\PY{l+s+s1}{\PYZsq{}}\PY{l+s+s1}{Speaker}\PY{l+s+s1}{\PYZsq{}}\PY{p}{,}\PY{l+s+s1}{\PYZsq{}}\PY{l+s+s1}{Text}\PY{l+s+s1}{\PYZsq{}}\PY{p}{,}\PY{l+s+s1}{\PYZsq{}}\PY{l+s+s1}{Play}\PY{l+s+s1}{\PYZsq{}}\PY{p}{]}\PY{p}{]}
\end{Verbatim}
\end{tcolorbox}

    It looks like there are 125 rows which are missing text. I'm not sure
why these rows are missing, but there's no way to perform text analysis
on missing text, so I'll just drop those.

    \begin{tcolorbox}[breakable, size=fbox, boxrule=1pt, pad at break*=1mm,colback=cellbackground, colframe=cellborder]
\prompt{In}{incolor}{54}{\boxspacing}
\begin{Verbatim}[commandchars=\\\{\}]
\PY{n}{text\PYZus{}play2} \PY{o}{=} \PY{n}{text\PYZus{}play2}\PY{o}{.}\PY{n}{dropna}\PY{p}{(}\PY{p}{)}
\end{Verbatim}
\end{tcolorbox}

    Next, I concatenate the text so all the dialogue from the same character
in the same play becomes a single string instead of several strings.

    \begin{tcolorbox}[breakable, size=fbox, boxrule=1pt, pad at break*=1mm,colback=cellbackground, colframe=cellborder]
\prompt{In}{incolor}{55}{\boxspacing}
\begin{Verbatim}[commandchars=\\\{\}]
\PY{n}{text\PYZus{}play2}\PY{p}{[}\PY{l+s+s1}{\PYZsq{}}\PY{l+s+s1}{text\PYZus{}concat}\PY{l+s+s1}{\PYZsq{}}\PY{p}{]} \PY{o}{=} \PY{n}{text\PYZus{}play2}\PY{o}{.}\PY{n}{groupby}\PY{p}{(}\PY{p}{[}\PY{l+s+s1}{\PYZsq{}}\PY{l+s+s1}{Speaker}\PY{l+s+s1}{\PYZsq{}}\PY{p}{,}\PY{l+s+s1}{\PYZsq{}}\PY{l+s+s1}{Play}\PY{l+s+s1}{\PYZsq{}}\PY{p}{]}\PY{p}{)}\PY{p}{[}\PY{l+s+s1}{\PYZsq{}}\PY{l+s+s1}{Text}\PY{l+s+s1}{\PYZsq{}}\PY{p}{]}\PY{o}{.}\PY{n}{transform}\PY{p}{(}\PY{k}{lambda} \PY{n}{x}\PY{p}{:} \PY{l+s+s1}{\PYZsq{}}\PY{l+s+s1}{,}\PY{l+s+s1}{\PYZsq{}}\PY{o}{.}\PY{n}{join}\PY{p}{(}\PY{n}{x}\PY{p}{)}\PY{p}{)}
\end{Verbatim}
\end{tcolorbox}

    Next, I drop duplicates, so each Speaker-Play pair only appears once in
the dataset.

    \begin{tcolorbox}[breakable, size=fbox, boxrule=1pt, pad at break*=1mm,colback=cellbackground, colframe=cellborder]
\prompt{In}{incolor}{57}{\boxspacing}
\begin{Verbatim}[commandchars=\\\{\}]
\PY{n}{text\PYZus{}play3} \PY{o}{=} \PY{n}{text\PYZus{}play2}\PY{o}{.}\PY{n}{drop\PYZus{}duplicates}\PY{p}{(}\PY{n}{subset}\PY{o}{=}\PY{p}{[}\PY{l+s+s1}{\PYZsq{}}\PY{l+s+s1}{Speaker}\PY{l+s+s1}{\PYZsq{}}\PY{p}{,}\PY{l+s+s1}{\PYZsq{}}\PY{l+s+s1}{Play}\PY{l+s+s1}{\PYZsq{}}\PY{p}{]}\PY{p}{)}
\end{Verbatim}
\end{tcolorbox}

    We've discussed its limitations in class, particularly with respect to
the nature of the tokenizer, but I choose to use scikit-learn to
implement TF-IDF (term frequency-inverse document frequency) for this
part. The scikit-learn tokenizer might not be effective for all document
types or text sources, but for this case, where the corpus consists of
standard English literature, the scikit-learn tokenizer is perfectly
adequate.

TF-IDF is a measure of term significance. It is proportional to the
frequency of a token in a document, and inversely proportional to the
frequency of the term across all documents. A token like ``the'', which
is highly prevalent in many texts, would have a low TF-IDF score, even
though it is common. Conversely, a token like ``TFIDF'', which only
occurs in a small corpus of documents (namely, those regarding text
analysis and NLP) but is highly prevalent in those specific documents
might have a high TFIDF score.

    \begin{tcolorbox}[breakable, size=fbox, boxrule=1pt, pad at break*=1mm,colback=cellbackground, colframe=cellborder]
\prompt{In}{incolor}{127}{\boxspacing}
\begin{Verbatim}[commandchars=\\\{\}]
\PY{k+kn}{from} \PY{n+nn}{sklearn}\PY{n+nn}{.}\PY{n+nn}{feature\PYZus{}extraction}\PY{n+nn}{.}\PY{n+nn}{text} \PY{k+kn}{import} \PY{n}{TfidfVectorizer}\PY{p}{,} \PY{n}{TfidfTransformer}\PY{p}{,} \PY{n}{CountVectorizer}
\end{Verbatim}
\end{tcolorbox}

    Now I instantiate a vectorizer, fit it to the series of dialogue
strings, and then convert it to a data frame.

    \begin{tcolorbox}[breakable, size=fbox, boxrule=1pt, pad at break*=1mm,colback=cellbackground, colframe=cellborder]
\prompt{In}{incolor}{157}{\boxspacing}
\begin{Verbatim}[commandchars=\\\{\}]
\PY{n}{vect} \PY{o}{=} \PY{n}{TfidfVectorizer}\PY{p}{(}\PY{p}{)}
\PY{n}{tfidf\PYZus{}matrix} \PY{o}{=} \PY{n}{vect}\PY{o}{.}\PY{n}{fit\PYZus{}transform}\PY{p}{(}\PY{n}{text\PYZus{}play3}\PY{p}{[}\PY{l+s+s1}{\PYZsq{}}\PY{l+s+s1}{text\PYZus{}concat}\PY{l+s+s1}{\PYZsq{}}\PY{p}{]}\PY{p}{)}
\PY{n}{df} \PY{o}{=} \PY{n}{pd}\PY{o}{.}\PY{n}{DataFrame}\PY{p}{(}\PY{n}{tfidf\PYZus{}matrix}\PY{o}{.}\PY{n}{toarray}\PY{p}{(}\PY{p}{)}\PY{p}{,} \PY{n}{columns} \PY{o}{=} \PY{n}{vect}\PY{o}{.}\PY{n}{get\PYZus{}feature\PYZus{}names}\PY{p}{(}\PY{p}{)}\PY{p}{)}
\end{Verbatim}
\end{tcolorbox}

    I calculate the mean TF-IDF score for each Speaker-Play pair:

    \begin{tcolorbox}[breakable, size=fbox, boxrule=1pt, pad at break*=1mm,colback=cellbackground, colframe=cellborder]
\prompt{In}{incolor}{158}{\boxspacing}
\begin{Verbatim}[commandchars=\\\{\}]
\PY{n}{df}\PY{p}{[}\PY{l+s+s1}{\PYZsq{}}\PY{l+s+s1}{mean\PYZus{}TFIDF}\PY{l+s+s1}{\PYZsq{}}\PY{p}{]} \PY{o}{=} \PY{n}{df}\PY{o}{.}\PY{n}{mean}\PY{p}{(}\PY{n}{axis} \PY{o}{=} \PY{l+m+mi}{1}\PY{p}{)}
\end{Verbatim}
\end{tcolorbox}

    Then I add in the speaker and play as identifiers, and select them and
the means.

    \begin{tcolorbox}[breakable, size=fbox, boxrule=1pt, pad at break*=1mm,colback=cellbackground, colframe=cellborder]
\prompt{In}{incolor}{159}{\boxspacing}
\begin{Verbatim}[commandchars=\\\{\}]
\PY{n}{df\PYZus{}TFIDF} \PY{o}{=} \PY{n}{pd}\PY{o}{.}\PY{n}{concat}\PY{p}{(}\PY{p}{[}\PY{n}{text\PYZus{}play3}\PY{p}{[}\PY{p}{[}\PY{l+s+s1}{\PYZsq{}}\PY{l+s+s1}{Speaker}\PY{l+s+s1}{\PYZsq{}}\PY{p}{,}\PY{l+s+s1}{\PYZsq{}}\PY{l+s+s1}{Play}\PY{l+s+s1}{\PYZsq{}}\PY{p}{]}\PY{p}{]}\PY{o}{.}\PY{n}{reset\PYZus{}index}\PY{p}{(}\PY{n}{drop}\PY{o}{=}\PY{k+kc}{True}\PY{p}{)}\PY{p}{,} \PY{n}{df}\PY{p}{]}\PY{p}{,} \PY{n}{axis}\PY{o}{=}\PY{l+m+mi}{1}\PY{p}{)}
\end{Verbatim}
\end{tcolorbox}

    \begin{tcolorbox}[breakable, size=fbox, boxrule=1pt, pad at break*=1mm,colback=cellbackground, colframe=cellborder]
\prompt{In}{incolor}{161}{\boxspacing}
\begin{Verbatim}[commandchars=\\\{\}]
\PY{n}{df\PYZus{}TFIDF} \PY{o}{=} \PY{n}{df\PYZus{}TFIDF}\PY{p}{[}\PY{p}{[}\PY{l+s+s1}{\PYZsq{}}\PY{l+s+s1}{Speaker}\PY{l+s+s1}{\PYZsq{}}\PY{p}{,}\PY{l+s+s1}{\PYZsq{}}\PY{l+s+s1}{Play}\PY{l+s+s1}{\PYZsq{}}\PY{p}{,}\PY{l+s+s1}{\PYZsq{}}\PY{l+s+s1}{mean\PYZus{}TFIDF}\PY{l+s+s1}{\PYZsq{}}\PY{p}{]}\PY{p}{]}
\end{Verbatim}
\end{tcolorbox}

    Calculating the mean TF-IDF score for each character-play pair will show
who uses the most unusual or distinctive words in this corpus of
Shakespeare plays.

    \begin{tcolorbox}[breakable, size=fbox, boxrule=1pt, pad at break*=1mm,colback=cellbackground, colframe=cellborder]
\prompt{In}{incolor}{181}{\boxspacing}
\begin{Verbatim}[commandchars=\\\{\}]
\PY{n}{df\PYZus{}TFIDF}\PY{o}{.}\PY{n}{sort\PYZus{}values}\PY{p}{(}\PY{p}{[}\PY{l+s+s1}{\PYZsq{}}\PY{l+s+s1}{mean\PYZus{}TFIDF}\PY{l+s+s1}{\PYZsq{}}\PY{p}{]}\PY{p}{,} \PY{n}{ascending} \PY{o}{=} \PY{k+kc}{False}\PY{p}{)}\PY{o}{.}\PY{n}{head}\PY{p}{(}\PY{p}{)}
\end{Verbatim}
\end{tcolorbox}

            \begin{tcolorbox}[breakable, size=fbox, boxrule=.5pt, pad at break*=1mm, opacityfill=0]
\prompt{Out}{outcolor}{181}{\boxspacing}
\begin{Verbatim}[commandchars=\\\{\}]
             Speaker               Play  mean\_TFIDF
301          Leontes  The Winter's Tale    0.001835
157  King Richard Ii    King Richard II    0.001829
217            Romeo   Romeo and Juliet    0.001822
221         Mercutio   Romeo and Juliet    0.001781
26        Coriolanus         Coriolanus    0.001781
\end{Verbatim}
\end{tcolorbox}
        
    The character Leontes, from the play The Winter's Tale, has the highest
TF-IDF score. Granted, this, and the other analyses done in this part,
are predicated on the idea that the dialogue was appropriately matched
to a play in part 2 (though we know that this line was spoken by
Leontes, since the speaker identifiers were included in the original
data).

    \begin{tcolorbox}[breakable, size=fbox, boxrule=1pt, pad at break*=1mm,colback=cellbackground, colframe=cellborder]
\prompt{In}{incolor}{163}{\boxspacing}
\begin{Verbatim}[commandchars=\\\{\}]
\PY{n}{df\PYZus{}TFIDF}\PY{p}{[}\PY{n}{df\PYZus{}TFIDF}\PY{p}{[}\PY{l+s+s1}{\PYZsq{}}\PY{l+s+s1}{mean\PYZus{}TFIDF}\PY{l+s+s1}{\PYZsq{}}\PY{p}{]}\PY{o}{==}\PY{n}{df\PYZus{}TFIDF}\PY{p}{[}\PY{l+s+s1}{\PYZsq{}}\PY{l+s+s1}{mean\PYZus{}TFIDF}\PY{l+s+s1}{\PYZsq{}}\PY{p}{]}\PY{o}{.}\PY{n}{min}\PY{p}{(}\PY{p}{)}\PY{p}{]}
\end{Verbatim}
\end{tcolorbox}

            \begin{tcolorbox}[breakable, size=fbox, boxrule=.5pt, pad at break*=1mm, opacityfill=0]
\prompt{Out}{outcolor}{163}{\boxspacing}
\begin{Verbatim}[commandchars=\\\{\}]
            Speaker                     Play  mean\_TFIDF
58              All         Double Falsehood    0.000089
64    Both Tribunes              Richard III    0.000089
167             All          King Richard II    0.000089
331  Duke Vincentio        The Winter's Tale    0.000089
365             All  The Taming of the Shrew    0.000089
392          Master  The Taming of the Shrew    0.000089
\end{Verbatim}
\end{tcolorbox}
        
    There is a six-way tie for lowest mean TF-IDF score. Notably, four of
these speakers are ensembles: the two ``All''s and ``Both Tribunes''.
This likely reflects the fact that in many Shakespeare plays, the
ensemble dialogue mainly consists of short exclamations or invocations
for main characters to deliver monologues.

    \begin{tcolorbox}[breakable, size=fbox, boxrule=1pt, pad at break*=1mm,colback=cellbackground, colframe=cellborder]
\prompt{In}{incolor}{168}{\boxspacing}
\begin{Verbatim}[commandchars=\\\{\}]
\PY{n}{tfidf\PYZus{}plays} \PY{o}{=} \PY{n}{df\PYZus{}TFIDF}\PY{o}{.}\PY{n}{groupby}\PY{p}{(}\PY{p}{[}\PY{l+s+s1}{\PYZsq{}}\PY{l+s+s1}{Play}\PY{l+s+s1}{\PYZsq{}}\PY{p}{]}\PY{p}{)}\PY{o}{.}\PY{n}{mean}\PY{p}{(}\PY{p}{[}\PY{l+s+s1}{\PYZsq{}}\PY{l+s+s1}{mean\PYZus{}TFIDF}\PY{l+s+s1}{\PYZsq{}}\PY{p}{]}\PY{p}{)}
\end{Verbatim}
\end{tcolorbox}

    \begin{tcolorbox}[breakable, size=fbox, boxrule=1pt, pad at break*=1mm,colback=cellbackground, colframe=cellborder]
\prompt{In}{incolor}{170}{\boxspacing}
\begin{Verbatim}[commandchars=\\\{\}]
\PY{n}{tfidf\PYZus{}plays}\PY{o}{.}\PY{n}{sort\PYZus{}values}\PY{p}{(}\PY{l+s+s1}{\PYZsq{}}\PY{l+s+s1}{mean\PYZus{}TFIDF}\PY{l+s+s1}{\PYZsq{}}\PY{p}{)}
\end{Verbatim}
\end{tcolorbox}

            \begin{tcolorbox}[breakable, size=fbox, boxrule=.5pt, pad at break*=1mm, opacityfill=0]
\prompt{Out}{outcolor}{170}{\boxspacing}
\begin{Verbatim}[commandchars=\\\{\}]
                         mean\_TFIDF
Play
Henry VIII                 0.000280
Titus Andronicus           0.000316
Double Falsehood           0.000503
Timon of Athens            0.000558
Twelfth Night              0.000626
Henry VI, Part 2           0.000679
Coriolanus                 0.000703
Sir Thomas More            0.000727
The Taming of the Shrew    0.000740
Richard III                0.000776
Henry VI, Part 1           0.000777
Measure for Measure        0.000851
Romeo and Juliet           0.000870
King Richard II            0.000898
Henry VI, Part 3           0.000912
The Tempest                0.000928
The Winter's Tale          0.000929
Edward III                 0.001009
Cymbeline                  0.001088
\end{Verbatim}
\end{tcolorbox}
        
    Based on these results (and again, assuming that dialogue was mostly
matched to the correct play), Cymbeline, Edward III, and The Winter's
Tale are among the most distinctive of Shakespeare's plays, while Henry
VIII, Titus Andronicus, and Double Falsehood are among the least
distinctive. I tried to identify a pattern based on the chronology of
the plays
\url{https://en.wikipedia.org/wiki/Chronology_of_Shakespeare\%27s_plays},
but none is evident; Cymbeline was one of Shakespeare's last plays, but
so was Henry VIII.

    \begin{tcolorbox}[breakable, size=fbox, boxrule=1pt, pad at break*=1mm,colback=cellbackground, colframe=cellborder]
\prompt{In}{incolor}{172}{\boxspacing}
\begin{Verbatim}[commandchars=\\\{\}]
\PY{n}{values} \PY{o}{=} \PY{n}{df\PYZus{}TFIDF}\PY{p}{[}\PY{l+s+s1}{\PYZsq{}}\PY{l+s+s1}{Play}\PY{l+s+s1}{\PYZsq{}}\PY{p}{]}\PY{o}{.}\PY{n}{value\PYZus{}counts}\PY{p}{(}\PY{p}{)}
\PY{n}{values} \PY{o}{=} \PY{n}{pd}\PY{o}{.}\PY{n}{DataFrame}\PY{p}{(}\PY{n}{values}\PY{p}{)}
\PY{n}{values}
\end{Verbatim}
\end{tcolorbox}

            \begin{tcolorbox}[breakable, size=fbox, boxrule=.5pt, pad at break*=1mm, opacityfill=0]
\prompt{Out}{outcolor}{172}{\boxspacing}
\begin{Verbatim}[commandchars=\\\{\}]
                         Play
Richard III                54
Romeo and Juliet           44
Coriolanus                 36
The Taming of the Shrew    35
The Winter's Tale          31
Henry VI, Part 3           30
King Richard II            29
Sir Thomas More            27
Measure for Measure        26
Henry VI, Part 1           17
Henry VI, Part 2           15
The Tempest                14
Cymbeline                  12
Timon of Athens            12
Twelfth Night              11
Double Falsehood            9
Titus Andronicus            3
Henry VIII                  1
Edward III                  1
\end{Verbatim}
\end{tcolorbox}
        
    There also does not appear to be a significant relationship between the
number of characters identified as being part of a particular play and
the mean TF-IDF score of that play. However, the play Edward III had one
of the highest mean TF-IDF scores, and only one character from that play
is identified in this data set.

    \begin{tcolorbox}[breakable, size=fbox, boxrule=1pt, pad at break*=1mm,colback=cellbackground, colframe=cellborder]
\prompt{In}{incolor}{173}{\boxspacing}
\begin{Verbatim}[commandchars=\\\{\}]
\PY{n}{text\PYZus{}play3}\PY{p}{[}\PY{n}{text\PYZus{}play3}\PY{p}{[}\PY{l+s+s1}{\PYZsq{}}\PY{l+s+s1}{Play}\PY{l+s+s1}{\PYZsq{}}\PY{p}{]}\PY{o}{==}\PY{l+s+s1}{\PYZsq{}}\PY{l+s+s1}{Edward III}\PY{l+s+s1}{\PYZsq{}}\PY{p}{]}
\end{Verbatim}
\end{tcolorbox}

            \begin{tcolorbox}[breakable, size=fbox, boxrule=.5pt, pad at break*=1mm, opacityfill=0]
\prompt{Out}{outcolor}{173}{\boxspacing}
\begin{Verbatim}[commandchars=\\\{\}]
      Line\_Number        Speaker  \textbackslash{}
2258         1568  Prince Edward

                                           Text  \textbackslash{}
2258  No, uncle; but our crosses on the way\textbackslash{}nHave made it tedious, wearisome,
and heavy\textbackslash{}nI want more uncles here to welcome me.

            Play  \textbackslash{}
2258  Edward III

                                     text\_concat
2258  No, uncle; but our crosses on the way\textbackslash{}nHave made it tedious, wearisome,
and heavy\textbackslash{}nI want more uncles here to welcome me.,God keep me from false
friends! but they were none.,Methinks a woman of this valiant spirit\textbackslash{}nShould, if
a coward heard her speak these words,\textbackslash{}nInfuse his breast with magnanimity\textbackslash{}nAnd
make him, naked, foil a man at arms.\textbackslash{}nI speak not this as doubting any here\textbackslash{}nFor
did I but suspect a fearful man\textbackslash{}nHe should have leave to go away betimes,\textbackslash{}nLest
in our need he might infect another\textbackslash{}nAnd make him of like spirit to himself.\textbackslash{}nIf
any such be here--as God forbid!--\textbackslash{}nLet him depart before we need his help.,And
take his thanks that yet hath nothing else.,Speak like a subject, proud
ambitious York!\textbackslash{}nSuppose that I am now my father's mouth;\textbackslash{}nResign thy chair, and
where I stand kneel thou,\textbackslash{}nWhilst I propose the selfsame words to thee,\textbackslash{}nWhich
traitor, thou wouldst have me answer to.,Let AEsop fable in a winter's
night;\textbackslash{}nHis currish riddles sort not with this place.,Nay, take away this
scolding crookback rather.,I know my duty; you are all undutiful:\textbackslash{}nLascivious
Edward, and thou perjured George,\textbackslash{}nAnd thou mis-shapen Dick, I tell ye all\textbackslash{}nI am
your better, traitors as ye are:\textbackslash{}nAnd thou usurp'st my father's right and mine.
\end{Verbatim}
\end{tcolorbox}
        
    That character is Prince Edward (listed on Wikipedia as Edward the Black
Prince), the eldest son of King Edward III and the heir apparent to the
throne until his death in 1376. Just looking at this dialogue, several
unusual words stand out: magnanimity, valiant, selfsame, currish,
crookback, lascivious, perjured and usurp'st all make appearances in
this text. This is likely an excerpt from a dramatic monologue, maybe
the one where he almost loses in battle to the French. This might be
going a bit too deep down the Shakespeare rabbit hole, but the Wikipedia
page for Edward III presents theories and research by Shakespeare
experts which cast doubt on the idea that he even wrote this play, or at
least that he was the sole author.

    Having acknowledged this, it is necessary to point out that Prince
Edward does not rank in the top 100 characters in this data set for
highest mean TF-IDF score. Across all the plays included, there are 407
characters; as such, while Prince Edward is in the top 25\%, he is by no
means an exception with respect to unusual vocabulary. Examining the
character with the most distinctive dialogue, Leontes, is revealing:

    \begin{tcolorbox}[breakable, size=fbox, boxrule=1pt, pad at break*=1mm,colback=cellbackground, colframe=cellborder]
\prompt{In}{incolor}{182}{\boxspacing}
\begin{Verbatim}[commandchars=\\\{\}]
\PY{n}{text\PYZus{}play3}\PY{p}{[}\PY{n}{text\PYZus{}play3}\PY{p}{[}\PY{l+s+s1}{\PYZsq{}}\PY{l+s+s1}{Speaker}\PY{l+s+s1}{\PYZsq{}}\PY{p}{]}\PY{o}{==}\PY{l+s+s1}{\PYZsq{}}\PY{l+s+s1}{Leontes}\PY{l+s+s1}{\PYZsq{}}\PY{p}{]}
\end{Verbatim}
\end{tcolorbox}

            \begin{tcolorbox}[breakable, size=fbox, boxrule=.5pt, pad at break*=1mm, opacityfill=0]
\prompt{Out}{outcolor}{182}{\boxspacing}
\begin{Verbatim}[commandchars=\\\{\}]
      Line\_Number  Speaker  \textbackslash{}
5914         4418  Leontes

                                                        Text  \textbackslash{}
5914  Stay your thanks a while;\textbackslash{}nAnd pay them when you part.

                   Play  \textbackslash{}
5914  The Winter's Tale

                                     text\_concat
5914  Stay your thanks a while;\textbackslash{}nAnd pay them when you part.,We are tougher,
brother,\textbackslash{}nThan you can put us to't.,One seven-night longer.,We'll part the time
between's then; and in that\textbackslash{}nI'll no gainsaying.,Tongue-tied, our queen?\textbackslash{}nspeak
you.,Well said, Hermione.,Is he won yet?,At my request he would not.\textbackslash{}nHermione,
my dearest, thou never spokest\textbackslash{}nTo better purpose.,Never, but once.,Why, that
was when\textbackslash{}nThree crabbed months had sour'd themselves to death,\textbackslash{}nEre I could make
thee open thy white hand\textbackslash{}nAnd clap thyself my love: then didst thou utter\textbackslash{}n'I am
yours for ever.',I' fecks!\textbackslash{}nWhy, that's my bawcock. What, hast\textbackslash{}nsmutch'd thy
nose?\textbackslash{}nThey say it is a copy out of mine. Come, captain,\textbackslash{}nWe must be neat; not
neat, but cleanly, captain:\textbackslash{}nAnd yet the steer, the heifer and the calf\textbackslash{}nAre all
call'd neat.--Still virginalling\textbackslash{}nUpon his palm!--How now, you wanton calf!\textbackslash{}nArt
thou my calf?,Thou want'st a rough pash and the shoots that I have,\textbackslash{}nTo be full
like me: yet they say we are\textbackslash{}nAlmost as like as eggs; women say so,\textbackslash{}nThat will
say anything but were they false\textbackslash{}nAs o'er-dyed blacks, as wind, as waters,
false\textbackslash{}nAs dice are to be wish'd by one that fixes\textbackslash{}nNo bourn 'twixt his and mine,
yet were it true\textbackslash{}nTo say this boy were like me. Come, sir page,\textbackslash{}nLook on me with
your welkin eye: sweet villain!\textbackslash{}nMost dear'st! my collop! Can thy dam?--may't
be?--\textbackslash{}nAffection! thy intention stabs the centre:\textbackslash{}nThou dost make possible
things not so held,\textbackslash{}nCommunicatest with dreams;--how can this be?--\textbackslash{}nWith what's
unreal thou coactive art,\textbackslash{}nAnd fellow'st nothing: then 'tis very credent\textbackslash{}nThou
mayst co-join with something; and thou dost,\textbackslash{}nAnd that beyond commission, and I
find it,\textbackslash{}nAnd that to the infection of my brains\textbackslash{}nAnd hardening of my brows.,No,
in good earnest.\textbackslash{}nHow sometimes nature will betray its folly,\textbackslash{}nIts tenderness,
and make itself a pastime\textbackslash{}nTo harder bosoms! Looking on the lines\textbackslash{}nOf my boy's
face, methoughts I did recoil\textbackslash{}nTwenty-three years, and saw myself
unbreech'd,\textbackslash{}nIn my green velvet coat, my dagger muzzled,\textbackslash{}nLest it should bite
its master, and so prove,\textbackslash{}nAs ornaments oft do, too dangerous:\textbackslash{}nHow like,
methought, I then was to this kernel,\textbackslash{}nThis squash, this gentleman. Mine honest
friend,\textbackslash{}nWill you take eggs for money?,You will! why, happy man be's dole! My
brother,\textbackslash{}nAre you so fond of your young prince as we\textbackslash{}nDo seem to be of ours?,So
stands this squire\textbackslash{}nOfficed with me: we two will walk, my lord,\textbackslash{}nAnd leave you
to your graver steps. Hermione,\textbackslash{}nHow thou lovest us, show in our brother's
welcome;\textbackslash{}nLet what is dear in Sicily be cheap:\textbackslash{}nNext to thyself and my young
rover, he's\textbackslash{}nApparent to my heart.,To your own bents dispose you: you'll be
found,\textbackslash{}nBe you beneath the sky.\textbackslash{}nI am angling now,\textbackslash{}nThough you perceive me not
how I give line.\textbackslash{}nGo to, go to!\textbackslash{}nHow she holds up the neb, the bill to him!\textbackslash{}nAnd
arms her with the boldness of a wife\textbackslash{}nTo her allowing husband!\textbackslash{}nGone
already!\textbackslash{}nInch-thick, knee-deep, o'er head and\textbackslash{}nears a fork'd one!\textbackslash{}nGo, play,
boy, play: thy mother plays, and I\textbackslash{}nPlay too, but so disgraced a part, whose
issue\textbackslash{}nWill hiss me to my grave: contempt and clamour\textbackslash{}nWill be my knell. Go,
play, boy, play.\textbackslash{}nThere have been,\textbackslash{}nOr I am much deceived, cuckolds ere
now;\textbackslash{}nAnd many a man there is, even at this present,\textbackslash{}nNow while I speak this,
holds his wife by the arm,\textbackslash{}nThat little thinks she has been sluiced in's
absence\textbackslash{}nAnd his pond fish'd by his next neighbour, by\textbackslash{}nSir Smile, his
neighbour: nay, there's comfort in't\textbackslash{}nWhiles other men have gates and those
gates open'd,\textbackslash{}nAs mine, against their will. Should all despair\textbackslash{}nThat have
revolted wives, the tenth of mankind\textbackslash{}nWould hang themselves. Physic for't there
is none;\textbackslash{}nIt is a bawdy planet, that will strike\textbackslash{}nWhere 'tis predominant; and
'tis powerful, think it,\textbackslash{}nFrom east, west, north and south: be it concluded,\textbackslash{}nNo
barricado for a belly; know't;\textbackslash{}nIt will let in and out the enemy\textbackslash{}nWith bag and
baggage: many thousand on's\textbackslash{}nHave the disease, and feel't not. How now, boy!,Why
that's some comfort. What, Camillo there?,Go play, Mamillius; thou'rt an honest
man.\textbackslash{}nCamillo, this great sir will yet stay longer.,Didst note it?,Didst
perceive it?\textbackslash{}nThey're here with me already, whispering, rounding\textbackslash{}n'Sicilia is a
so-forth:' 'tis far gone,\textbackslash{}nWhen I shall gust it last. How came't, Camillo,\textbackslash{}nThat
he did stay?,At the queen's be't: 'good' should be pertinent\textbackslash{}nBut, so it is, it
is not. Was this taken\textbackslash{}nBy any understanding pate but thine?\textbackslash{}nFor thy conceit is
soaking, will draw in\textbackslash{}nMore than the common blocks: not noted, is't,\textbackslash{}nBut of the
finer natures? by some severals\textbackslash{}nOf head-piece extraordinary? lower
messes\textbackslash{}nPerchance are to this business purblind? say.,Ha!,Ay, but
why?,Satisfy!\textbackslash{}nThe entreaties of your mistress! satisfy!\textbackslash{}nLet that suffice. I
have trusted thee, Camillo,\textbackslash{}nWith all the nearest things to my heart, as
well\textbackslash{}nMy chamber-councils, wherein, priest-like, thou\textbackslash{}nHast cleansed my bosom, I
from thee departed\textbackslash{}nThy penitent reform'd: but we have been\textbackslash{}nDeceived in thy
integrity, deceived\textbackslash{}nIn that which seems so.,To bide upon't, thou art not
honest, or,\textbackslash{}nIf thou inclinest that way, thou art a coward,\textbackslash{}nWhich hoxes honesty
behind, restraining\textbackslash{}nFrom course required; or else thou must be counted\textbackslash{}nA
servant grafted in my serious trust\textbackslash{}nAnd therein negligent; or else a fool\textbackslash{}nThat
seest a game play'd home, the rich stake drawn,\textbackslash{}nAnd takest it all for jest.,Ha'
not you seen, Camillo,--\textbackslash{}nBut that's past doubt, you have, or your eye-glass\textbackslash{}nIs
thicker than a cuckold's horn,--or heard,--\textbackslash{}nFor to a vision so apparent
rumour\textbackslash{}nCannot be mute,--or thought,--for cogitation\textbackslash{}nResides not in that man
that does not think,--\textbackslash{}nMy wife is slippery? If thou wilt confess,\textbackslash{}nOr else be
impudently negative,\textbackslash{}nTo have nor eyes nor ears nor thought, then say\textbackslash{}nMy wife's
a hobby-horse, deserves a name\textbackslash{}nAs rank as any flax-wench that puts to\textbackslash{}nBefore
her troth-plight: say't and justify't.,Is whispering nothing?\textbackslash{}nIs leaning cheek
to cheek? is meeting noses?\textbackslash{}nKissing with inside lip? stopping the career\textbackslash{}nOf
laughing with a sigh?--a note infallible\textbackslash{}nOf breaking honesty--horsing foot on
foot?\textbackslash{}nSkulking in corners? wishing clocks more swift?\textbackslash{}nHours, minutes? noon,
midnight? and all eyes\textbackslash{}nBlind with the pin and web but theirs, theirs
only,\textbackslash{}nThat would unseen be wicked? is this nothing?\textbackslash{}nWhy, then the world and
all that's in't is nothing;\textbackslash{}nThe covering sky is nothing; Bohemia nothing;\textbackslash{}nMy
wife is nothing; nor nothing have these nothings,\textbackslash{}nIf this be nothing.,Say it
be, 'tis true.,It is; you lie, you lie:\textbackslash{}nI say thou liest, Camillo, and I hate
thee,\textbackslash{}nPronounce thee a gross lout, a mindless slave,\textbackslash{}nOr else a hovering
temporizer, that\textbackslash{}nCanst with thine eyes at once see good and evil,\textbackslash{}nInclining to
them both: were my wife's liver\textbackslash{}nInfected as her life, she would not live\textbackslash{}nThe
running of one glass.,Why, he that wears her like a medal, hanging\textbackslash{}nAbout his
neck, Bohemia: who, if I\textbackslash{}nHad servants true about me, that bare eyes\textbackslash{}nTo see
alike mine honour as their profits,\textbackslash{}nTheir own particular thrifts, they would do
that\textbackslash{}nWhich should undo more doing: ay, and thou,\textbackslash{}nHis cupbearer,--whom I from
meaner form\textbackslash{}nHave benched and reared to worship, who mayst see\textbackslash{}nPlainly as
heaven sees earth and earth sees heaven,\textbackslash{}nHow I am galled,--mightst bespice a
cup,\textbackslash{}nTo give mine enemy a lasting wink;\textbackslash{}nWhich draught to me were cordial.,Make
that thy question, and go rot!\textbackslash{}nDost think I am so muddy, so unsettled,\textbackslash{}nTo
appoint myself in this vexation, sully\textbackslash{}nThe purity and whiteness of my
sheets,\textbackslash{}nWhich to preserve is sleep, which being spotted\textbackslash{}nIs goads, thorns,
nettles, tails of wasps,\textbackslash{}nGive scandal to the blood o' the prince my son,\textbackslash{}nWho I
do think is mine and love as mine,\textbackslash{}nWithout ripe moving to't? Would I do
this?\textbackslash{}nCould man so blench?,Thou dost advise me\textbackslash{}nEven so as I mine own course
have set down:\textbackslash{}nI'll give no blemish to her honour, none.,This is all:\textbackslash{}nDo't and
thou hast the one half of my heart;\textbackslash{}nDo't not, thou split'st thine own.,I will
seem friendly, as thou hast advised me.,Was he met there? his train? Camillo
with him?,How blest am I\textbackslash{}nIn my just censure, in my true opinion!\textbackslash{}nAlack, for
lesser knowledge! how accursed\textbackslash{}nIn being so blest! There may be in the cup\textbackslash{}nA
spider steep'd, and one may drink, depart,\textbackslash{}nAnd yet partake no venom, for his
knowledge\textbackslash{}nIs not infected: but if one present\textbackslash{}nThe abhorr'd ingredient to his
eye, make known\textbackslash{}nHow he hath drunk, he cracks his gorge, his sides,\textbackslash{}nWith
violent hefts. I have drunk,\textbackslash{}nand seen the spider.\textbackslash{}nCamillo was his help in
this, his pander:\textbackslash{}nThere is a plot against my life, my crown;\textbackslash{}nAll's true that
is mistrusted: that false villain\textbackslash{}nWhom I employ'd was pre-employ'd by him:\textbackslash{}nHe
has discover'd my design, and I\textbackslash{}nRemain a pinch'd thing; yea, a very trick\textbackslash{}nFor
them to play at will. How came the posterns\textbackslash{}nSo easily open?,I know't too
well.\textbackslash{}nGive me the boy: I am glad you did not nurse him:\textbackslash{}nThough he does bear
some signs of me, yet you\textbackslash{}nHave too much blood in him.,Bear the boy hence; he
shall not come about her;\textbackslash{}nAway with him! and let her sport herself\textbackslash{}nWith that
she's big with; for 'tis Polixenes\textbackslash{}nHas made thee swell thus.,You, my
lords,\textbackslash{}nLook on her, mark her well; be but about\textbackslash{}nTo say 'she is a goodly lady,'
and\textbackslash{}nThe justice of your bearts will thereto add\textbackslash{}n'Tis pity she's not honest,
honourable:'\textbackslash{}nPraise her but for this her without-door form,\textbackslash{}nWhich on my faith
deserves high speech, and straight\textbackslash{}nThe shrug, the hum or ha, these petty
brands\textbackslash{}nThat calumny doth use--O, I am out--\textbackslash{}nThat mercy does, for calumny will
sear\textbackslash{}nVirtue itself: these shrugs, these hums and ha's,\textbackslash{}nWhen you have said
'she's goodly,' come between\textbackslash{}nEre you can say 'she's honest:' but be 't
known,\textbackslash{}nFrom him that has most cause to grieve it should be,\textbackslash{}nShe's an
adulteress.,You have mistook, my lady,\textbackslash{}nPolixenes for Leontes: O thou
thing!\textbackslash{}nWhich I'll not call a creature of thy place,\textbackslash{}nLest barbarism, making me
the precedent,\textbackslash{}nShould a like language use to all degrees\textbackslash{}nAnd mannerly
distinguishment leave out\textbackslash{}nBetwixt the prince and beggar: I have said\textbackslash{}nShe's an
adulteress; I have said with whom:\textbackslash{}nMore, she's a traitor and Camillo is\textbackslash{}nA
federary with her, and one that knows\textbackslash{}nWhat she should shame to know
herself\textbackslash{}nBut with her most vile principal, that she's\textbackslash{}nA bed-swerver, even as
bad as those\textbackslash{}nThat vulgars give bold'st titles, ay, and privy\textbackslash{}nTo this their
late escape.,No; if I mistake\textbackslash{}nIn those foundations which I build upon,\textbackslash{}nThe
centre is not big enough to bear\textbackslash{}nA school-boy's top. Away with her! to
prison!\textbackslash{}nHe who shall speak for her is afar off guilty\textbackslash{}nBut that he
speaks.,Shall I be heard?,Go, do our bidding; hence!,Hold your peaces.,Cease; no
more.\textbackslash{}nYou smell this business with a sense as cold\textbackslash{}nAs is a dead man's nose:
but I do see't and feel't\textbackslash{}nAs you feel doing thus; and see withal\textbackslash{}nThe
instruments that feel.,What! lack I credit?,Why, what need we\textbackslash{}nCommune with you
of this, but rather follow\textbackslash{}nOur forceful instigation? Our prerogative\textbackslash{}nCalls not
your counsels, but our natural goodness\textbackslash{}nImparts this; which if you, or
stupefied\textbackslash{}nOr seeming so in skill, cannot or will not\textbackslash{}nRelish a truth like us,
inform yourselves\textbackslash{}nWe need no more of your advice: the matter,\textbackslash{}nThe loss, the
gain, the ordering on't, is all\textbackslash{}nProperly ours.,How could that be?\textbackslash{}nEither thou
art most ignorant by age,\textbackslash{}nOr thou wert born a fool. Camillo's flight,\textbackslash{}nAdded to
their familiarity,\textbackslash{}nWhich was as gross as ever touch'd conjecture,\textbackslash{}nThat lack'd
sight only, nought for approbation\textbackslash{}nBut only seeing, all other
circumstances\textbackslash{}nMade up to the deed, doth push on this proceeding:\textbackslash{}nYet, for a
greater confirmation,\textbackslash{}nFor in an act of this importance 'twere\textbackslash{}nMost piteous to
be wild, I have dispatch'd in post\textbackslash{}nTo sacred Delphos, to Apollo's
temple,\textbackslash{}nCleomenes and Dion, whom you know\textbackslash{}nOf stuff'd sufficiency: now from the
oracle\textbackslash{}nThey will bring all; whose spiritual counsel had,\textbackslash{}nShall stop or spur
me. Have I done well?,Though I am satisfied and need no more\textbackslash{}nThan what I know,
yet shall the oracle\textbackslash{}nGive rest to the minds of others, such as he\textbackslash{}nWhose
ignorant credulity will not\textbackslash{}nCome up to the truth. So have we thought it
good\textbackslash{}nFrom our free person she should be confined,\textbackslash{}nLest that the treachery of
the two fled hence\textbackslash{}nBe left her to perform. Come, follow us;\textbackslash{}nWe are to speak in
public; for this business\textbackslash{}nWill raise us all.,Nor night nor day no rest: it is
but weakness\textbackslash{}nTo bear the matter thus; mere weakness. If\textbackslash{}nThe cause were not in
being,--part o' the cause,\textbackslash{}nShe the adulteress; for the harlot king\textbackslash{}nIs quite
beyond mine arm, out of the blank\textbackslash{}nAnd level of my brain, plot-proof; but she\textbackslash{}nI
can hook to me: say that she were gone,\textbackslash{}nGiven to the fire, a moiety of my
rest\textbackslash{}nMight come to me again. Who's there?,How does the boy?,To see his
nobleness!\textbackslash{}nConceiving the dishonour of his mother,\textbackslash{}nHe straight declined,
droop'd, took it deeply,\textbackslash{}nFasten'd and fix'd the shame on't in himself,\textbackslash{}nThrew
off his spirit, his appetite, his sleep,\textbackslash{}nAnd downright languish'd. Leave me
solely: go,\textbackslash{}nSee how he fares.\textbackslash{}nFie, fie! no thought of him:\textbackslash{}nThe thought of my
revenges that way\textbackslash{}nRecoil upon me: in himself too mighty,\textbackslash{}nAnd in his parties,
his alliance; let him be\textbackslash{}nUntil a time may serve: for present vengeance,\textbackslash{}nTake
it on her. Camillo and Polixenes\textbackslash{}nLaugh at me, make their pastime at my
sorrow:\textbackslash{}nThey should not laugh if I could reach them, nor\textbackslash{}nShall she within my
power.,What noise there, ho?,How!\textbackslash{}nAway with that audacious lady! Antigonus,\textbackslash{}nI
charged thee that she should not come about me:\textbackslash{}nI knew she would.,What, canst
not rule her?,Good queen!,Force her hence.,Out!\textbackslash{}nA mankind witch! Hence with
her, out o' door:\textbackslash{}nA most intelligencing bawd!,Traitors!\textbackslash{}nWill you not push her
out? Give her the bastard.\textbackslash{}nThou dotard! thou art woman-tired, unroosted\textbackslash{}nBy thy
dame Partlet here. Take up the bastard;\textbackslash{}nTake't up, I say; give't to thy
crone.,He dreads his wife.,A nest of traitors!,A callat\textbackslash{}nOf boundless tongue,
who late hath beat her husband\textbackslash{}nAnd now baits me! This brat is none of mine;\textbackslash{}nIt
is the issue of Polixenes:\textbackslash{}nHence with it, and together with the dam\textbackslash{}nCommit
them to the fire!,A gross hag\textbackslash{}nAnd, lozel, thou art worthy to be hang'd,\textbackslash{}nThat
wilt not stay her tongue.,Once more, take her hence.,I'll ha' thee burnt.,On
your allegiance,\textbackslash{}nOut of the chamber with her! Were I a tyrant,\textbackslash{}nWhere were her
life? she durst not call me so,\textbackslash{}nIf she did know me one. Away with her!,Thou,
traitor, hast set on thy wife to this.\textbackslash{}nMy child? away with't! Even thou, that
hast\textbackslash{}nA heart so tender o'er it, take it hence\textbackslash{}nAnd see it instantly consumed
with fire;\textbackslash{}nEven thou and none but thou. Take it up straight:\textbackslash{}nWithin this hour
bring me word 'tis done,\textbackslash{}nAnd by good testimony, or I'll seize thy life,\textbackslash{}nWith
what thou else call'st thine. If thou refuse\textbackslash{}nAnd wilt encounter with my wrath,
say so;\textbackslash{}nThe bastard brains with these my proper hands\textbackslash{}nShall I dash out. Go,
take it to the fire;\textbackslash{}nFor thou set'st on thy wife.,You're liars all.,I am a
feather for each wind that blows:\textbackslash{}nShall I live on to see this bastard
kneel\textbackslash{}nAnd call me father? better burn it now\textbackslash{}nThan curse it then. But be it;
let it live.\textbackslash{}nIt shall not neither. You, sir, come you hither;\textbackslash{}nYou that have
been so tenderly officious\textbackslash{}nWith Lady Margery, your midwife there,\textbackslash{}nTo save this
bastard's life,--for 'tis a bastard,\textbackslash{}nSo sure as this beard's grey,\textbackslash{}n--what will
you adventure\textbackslash{}nTo save this brat's life?,It shall be possible. Swear by this
sword\textbackslash{}nThou wilt perform my bidding.,Mark and perform it, see'st thou! for the
fail\textbackslash{}nOf any point in't shall not only be\textbackslash{}nDeath to thyself but to thy lewd-
tongued wife,\textbackslash{}nWhom for this time we pardon. We enjoin thee,\textbackslash{}nAs thou art liege-
man to us, that thou carry\textbackslash{}nThis female bastard hence and that thou bear it\textbackslash{}nTo
some remote and desert place quite out\textbackslash{}nOf our dominions, and that there thou
leave it,\textbackslash{}nWithout more mercy, to its own protection\textbackslash{}nAnd favour of the climate.
As by strange fortune\textbackslash{}nIt came to us, I do in justice charge thee,\textbackslash{}nOn thy
soul's peril and thy body's torture,\textbackslash{}nThat thou commend it strangely to some
place\textbackslash{}nWhere chance may nurse or end it. Take it up.,No, I'll not
rear\textbackslash{}nAnother's issue.,Twenty-three days\textbackslash{}nThey have been absent: 'tis good
speed; foretells\textbackslash{}nThe great Apollo suddenly will have\textbackslash{}nThe truth of this appear.
Prepare you, lords;\textbackslash{}nSummon a session, that we may arraign\textbackslash{}nOur most disloyal
lady, for, as she hath\textbackslash{}nBeen publicly accused, so shall she have\textbackslash{}nA just and
open trial. While she lives\textbackslash{}nMy heart will be a burthen to me. Leave me,\textbackslash{}nAnd
think upon my bidding.,This sessions, to our great grief we pronounce,\textbackslash{}nEven
pushes 'gainst our heart: the party tried\textbackslash{}nThe daughter of a king, our wife, and
one\textbackslash{}nOf us too much beloved. Let us be clear'd\textbackslash{}nOf being tyrannous, since we so
openly\textbackslash{}nProceed in justice, which shall have due course,\textbackslash{}nEven to the guilt or
the purgation.\textbackslash{}nProduce the prisoner.,Read the indictment.,I ne'er heard
yet\textbackslash{}nThat any of these bolder vices wanted\textbackslash{}nLess impudence to gainsay what they
did\textbackslash{}nThan to perform it first.,You will not own it.,You knew of his departure,
as you know\textbackslash{}nWhat you have underta'en to do in's absence.,Your actions are my
dreams;\textbackslash{}nYou had a bastard by Polixenes,\textbackslash{}nAnd I but dream'd it. As you were past
all shame,--\textbackslash{}nThose of your fact are so--so past all truth:\textbackslash{}nWhich to deny
concerns more than avails; for as\textbackslash{}nThy brat hath been cast out, like to
itself,\textbackslash{}nNo father owning it,--which is, indeed,\textbackslash{}nMore criminal in thee than it,
--so thou\textbackslash{}nShalt feel our justice, in whose easiest passage\textbackslash{}nLook for no less
than death.,Break up the seals and read.,Hast thou read truth?,There is no truth
at all i' the oracle:\textbackslash{}nThe sessions shall proceed: this is mere falsehood.,What
is the business?,How! gone!,Apollo's angry; and the heavens themselves\textbackslash{}nDo
strike at my injustice.\textbackslash{}nHow now there!,Take her hence:\textbackslash{}nHer heart is but
o'ercharged; she will recover:\textbackslash{}nI have too much believed mine own
suspicion:\textbackslash{}nBeseech you, tenderly apply to her\textbackslash{}nSome remedies for life.\textbackslash{}nApollo,
pardon\textbackslash{}nMy great profaneness 'gainst thine oracle!\textbackslash{}nI'll reconcile me to
Polixenes,\textbackslash{}nNew woo my queen, recall the good Camillo,\textbackslash{}nWhom I proclaim a man of
truth, of mercy;\textbackslash{}nFor, being transported by my jealousies\textbackslash{}nTo bloody thoughts
and to revenge, I chose\textbackslash{}nCamillo for the minister to poison\textbackslash{}nMy friend
Polixenes: which had been done,\textbackslash{}nBut that the good mind of Camillo tardied\textbackslash{}nMy
swift command, though I with death and with\textbackslash{}nReward did threaten and encourage
him,\textbackslash{}nNot doing 't and being done: he, most humane\textbackslash{}nAnd fill'd with honour, to
my kingly guest\textbackslash{}nUnclasp'd my practise, quit his fortunes here,\textbackslash{}nWhich you knew
great, and to the hazard\textbackslash{}nOf all encertainties himself commended,\textbackslash{}nNo richer
than his honour: how he glisters\textbackslash{}nThorough my rust! and how his pity\textbackslash{}nDoes my
deeds make the blacker!,Go on, go on\textbackslash{}nThou canst not speak too much; I have
deserved\textbackslash{}nAll tongues to talk their bitterest.,Thou didst speak but well\textbackslash{}nWhen
most the truth; which I receive much better\textbackslash{}nThan to be pitied of thee. Prithee,
bring me\textbackslash{}nTo the dead bodies of my queen and son:\textbackslash{}nOne grave shall be for both:
upon them shall\textbackslash{}nThe causes of their death appear, unto\textbackslash{}nOur shame perpetual.
Once a day I'll visit\textbackslash{}nThe chapel where they lie, and tears shed there\textbackslash{}nShall be
my recreation: so long as nature\textbackslash{}nWill bear up with this exercise, so long\textbackslash{}nI
daily vow to use it. Come and lead me\textbackslash{}nUnto these sorrows.,Whilst I
remember\textbackslash{}nHer and her virtues, I cannot forget\textbackslash{}nMy blemishes in them, and so
still think of\textbackslash{}nThe wrong I did myself; which was so much,\textbackslash{}nThat heirless it
hath made my kingdom and\textbackslash{}nDestroy'd the sweet'st companion that e'er man\textbackslash{}nBred
his hopes out of.,I think so. Kill'd!\textbackslash{}nShe I kill'd! I did so: but thou strikest
me\textbackslash{}nSorely, to say I did; it is as bitter\textbackslash{}nUpon thy tongue as in my thought:
now, good now,\textbackslash{}nSay so but seldom.,Good Paulina,\textbackslash{}nWho hast the memory of
Hermione,\textbackslash{}nI know, in honour, O, that ever I\textbackslash{}nHad squared me to thy counsel!
then, even now,\textbackslash{}nI might have look'd upon my queen's full eyes,\textbackslash{}nHave taken
treasure from her lips--,Thou speak'st truth.\textbackslash{}nNo more such wives; therefore, no
wife: one worse,\textbackslash{}nAnd better used, would make her sainted spirit\textbackslash{}nAgain possess
her corpse, and on this stage,\textbackslash{}nWhere we're offenders now, appear soul-
vex'd,\textbackslash{}nAnd begin, 'Why to me?',She had; and would incense me\textbackslash{}nTo murder her I
married.,Stars, stars,\textbackslash{}nAnd all eyes else dead coals! Fear thou no wife;\textbackslash{}nI'll
have no wife, Paulina.,Never, Paulina; so be blest my spirit!,My true
Paulina,\textbackslash{}nWe shall not marry till thou bid'st us.,What with him? he comes
not\textbackslash{}nLike to his father's greatness: his approach,\textbackslash{}nSo out of circumstance and
sudden, tells us\textbackslash{}n'Tis not a visitation framed, but forced\textbackslash{}nBy need and
accident. What train?,His princess, say you, with him?,Go, Cleomenes;\textbackslash{}nYourself,
assisted with your honour'd friends,\textbackslash{}nBring them to our embracement. Still, 'tis
strange\textbackslash{}nHe thus should steal upon us.,Prithee, no more; cease; thou know'st\textbackslash{}nHe
dies to me again when talk'd of: sure,\textbackslash{}nWhen I shall see this gentleman, thy
speeches\textbackslash{}nWill bring me to consider that which may\textbackslash{}nUnfurnish me of reason. They
are come.\textbackslash{}nYour mother was most true to wedlock, prince;\textbackslash{}nFor she did print your
royal father off,\textbackslash{}nConceiving you: were I but twenty-one,\textbackslash{}nYour father's image
is so hit in you,\textbackslash{}nHis very air, that I should call you brother,\textbackslash{}nAs I did him,
and speak of something wildly\textbackslash{}nBy us perform'd before. Most dearly welcome!\textbackslash{}nAnd
your fair princess,--goddess!--O, alas!\textbackslash{}nI lost a couple, that 'twixt heaven and
earth\textbackslash{}nMight thus have stood begetting wonder as\textbackslash{}nYou, gracious couple, do: and
then I lost--\textbackslash{}nAll mine own folly--the society,\textbackslash{}nAmity too, of your brave
father, whom,\textbackslash{}nThough bearing misery, I desire my life\textbackslash{}nOnce more to look on
him.,O my brother,\textbackslash{}nGood gentleman! the wrongs I have done thee stir\textbackslash{}nAfresh
within me, and these thy offices,\textbackslash{}nSo rarely kind, are as interpreters\textbackslash{}nOf my
behind-hand slackness. Welcome hither,\textbackslash{}nAs is the spring to the earth. And hath
he too\textbackslash{}nExposed this paragon to the fearful usage,\textbackslash{}nAt least ungentle, of the
dreadful Neptune,\textbackslash{}nTo greet a man not worth her pains, much less\textbackslash{}nThe adventure
of her person?,Where the warlike Smalus,\textbackslash{}nThat noble honour'd lord, is fear'd
and loved?,The blessed gods\textbackslash{}nPurge all infection from our air whilst you\textbackslash{}nDo
climate here! You have a holy father,\textbackslash{}nA graceful gentleman; against whose
person,\textbackslash{}nSo sacred as it is, I have done sin:\textbackslash{}nFor which the heavens, taking
angry note,\textbackslash{}nHave left me issueless; and your father's blest,\textbackslash{}nAs he from heaven
merits it, with you\textbackslash{}nWorthy his goodness. What might I have been,\textbackslash{}nMight I a son
and daughter now have look'd on,\textbackslash{}nSuch goodly things as you!,Where's Bohemia?
speak.,Who? Camillo?,You are married?,My lord,\textbackslash{}nIs this the daughter of a
king?,That 'once' I see by your good father's speed\textbackslash{}nWill come on very slowly. I
am sorry,\textbackslash{}nMost sorry, you have broken from his liking\textbackslash{}nWhere you were tied in
duty, and as sorry\textbackslash{}nYour choice is not so rich in worth as beauty,\textbackslash{}nThat you
might well enjoy her.,Would he do so, I'ld beg your precious mistress,\textbackslash{}nWhich he
counts but a trifle.,I thought of her,\textbackslash{}nEven in these looks I made.\textbackslash{}nBut your
petition\textbackslash{}nIs yet unanswer'd. I will to your father:\textbackslash{}nYour honour not o'erthrown
by your desires,\textbackslash{}nI am friend to them and you: upon which errand\textbackslash{}nI now go
toward him; therefore follow me\textbackslash{}nAnd mark what way I make: come, good my lord.,O
grave and good Paulina, the great comfort\textbackslash{}nThat I have had of thee!,O
Paulina,\textbackslash{}nWe honour you with trouble: but we came\textbackslash{}nTo see the statue of our
queen: your gallery\textbackslash{}nHave we pass'd through, not without much content\textbackslash{}nIn many
singularities; but we saw not\textbackslash{}nThat which my daughter came to look upon,\textbackslash{}nThe
statue of her mother.,Her natural posture!\textbackslash{}nChide me, dear stone, that I may say
indeed\textbackslash{}nThou art Hermione; or rather, thou art she\textbackslash{}nIn thy not chiding, for she
was as tender\textbackslash{}nAs infancy and grace. But yet, Paulina,\textbackslash{}nHermione was not so much
wrinkled, nothing\textbackslash{}nSo aged as this seems.,As now she might have done,\textbackslash{}nSo much
to my good comfort, as it is\textbackslash{}nNow piercing to my soul. O, thus she stood,\textbackslash{}nEven
with such life of majesty, warm life,\textbackslash{}nAs now it coldly stands, when first I
woo'd her!\textbackslash{}nI am ashamed: does not the stone rebuke me\textbackslash{}nFor being more stone
than it? O royal piece,\textbackslash{}nThere's magic in thy majesty, which has\textbackslash{}nMy evils
conjured to remembrance and\textbackslash{}nFrom thy admiring daughter took the
spirits,\textbackslash{}nStanding like stone with thee.,Do not draw the curtain.,Let be, let
be.\textbackslash{}nWould I were dead, but that, methinks, already--\textbackslash{}nWhat was he that did make
it? See, my lord,\textbackslash{}nWould you not deem it breathed? and that those veins\textbackslash{}nDid
verily bear blood?,The fixture of her eye has motion in't,\textbackslash{}nAs we are mock'd
with art.,O sweet Paulina,\textbackslash{}nMake me to think so twenty years together!\textbackslash{}nNo
settled senses of the world can match\textbackslash{}nThe pleasure of that madness. Let 't
alone.,Do, Paulina;\textbackslash{}nFor this affliction has a taste as sweet\textbackslash{}nAs any cordial
comfort. Still, methinks,\textbackslash{}nThere is an air comes from her: what fine
chisel\textbackslash{}nCould ever yet cut breath? Let no man mock me,\textbackslash{}nFor I will kiss her.,No,
not these twenty years.,What you can make her do,\textbackslash{}nI am content to look on: what
to speak,\textbackslash{}nI am content to hear; for 'tis as easy\textbackslash{}nTo make her speak as
move.,Proceed:\textbackslash{}nNo foot shall stir.,O, she's warm!\textbackslash{}nIf this be magic, let it be
an art\textbackslash{}nLawful as eating.,O, peace, Paulina!\textbackslash{}nThou shouldst a husband take by my
consent,\textbackslash{}nAs I by thine a wife: this is a match,\textbackslash{}nAnd made between's by vows.
Thou hast found mine;\textbackslash{}nBut how, is to be question'd; for I saw her,\textbackslash{}nAs I
thought, dead, and have in vain said many\textbackslash{}nA prayer upon her grave. I'll not
seek far--\textbackslash{}nFor him, I partly know his mind--to find thee\textbackslash{}nAn honourable
husband. Come, Camillo,\textbackslash{}nAnd take her by the hand, whose worth and honesty\textbackslash{}nIs
richly noted and here justified\textbackslash{}nBy us, a pair of kings. Let's from this
place.\textbackslash{}nWhat! look upon my brother: both your pardons,\textbackslash{}nThat e'er I put between
your holy looks\textbackslash{}nMy ill suspicion. This is your son-in-law,\textbackslash{}nAnd son unto the
king, who, heavens directing,\textbackslash{}nIs troth-plight to your daughter. Good
Paulina,\textbackslash{}nLead us from hence, where we may leisurely\textbackslash{}nEach one demand an answer
to his part\textbackslash{}nPerform'd in this wide gap of time since first\textbackslash{}nWe were dissever'd:
hastily lead away.
\end{Verbatim}
\end{tcolorbox}
        
    The length of dialogue attributed to this character is significantly
longer than that attributed to Prince Edward. This reveals a potential
weakness of TF-IDF: difficulty handling different document sizes.
Leontes might not have a more interesting or idiosyncratic vocabulary
than Prince Edward; he might just talk more. Future steps for this
analysis could weight TF-IDF scores by document size, or could select a
random sample of tokens from each document to use to calculate scores.

That being said, I do think that TF-IDF is useful as a means of
evaluating terms and documents in relation to each other. This metric
makes it easy to identify the most distinctive terms and tokens in a
document, and also helps identify documents which could be outliers in
the corpus. In this case, I found that one of the plays with the highest
mean TF-IDF scores was potentially not written by Shakespeare!
Additionally, since the sole character we have data on in that play had
a TF-IDF score which was only in the top 30\% of characters in the data
set, this suggests that many other characters, like ensembles or parts
which only appear in one or two scenes, have relatively generic dialogue
or do not use unusual or uncommon words. TF-IDF can also be implemented
relatively easily, without having to download a large model or data set,
making it easier and quicker to use than other processing methods such
as embeddings.


    % Add a bibliography block to the postdoc
    
    
    
\end{document}
